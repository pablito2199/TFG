\section{Casos de uso}

Un caso de uso \cite{casodeuso} es cualquier acción o actividad que se realiza en el sistema. A continuación se presentan los diferentes casos de uso identificados entre las funcionalidades de la aplicación.

\begin{table}[H]
\begin{center}
\begin{tabular}{|p{3cm}|p{10cm}|} \hline
\centering {\bf UC-01} & Iniciar sesión  \\ \hline\hline
\centering {\bf Versión} & 1.0 (05/06/2022) \\ \hline
\centering {\bf Dependencias} &  FRQ-01. \\ \hline
\centering {\bf Descripción} &  El sistema deberá comportarse tal como se describe en el siguiente caso de uso cuando el usuario trate de iniciar sesión en la aplicación. \\ \hline
\centering {\bf Precondición} &  El usuario debe de estar dado de alta en el sistema. \\ \hline
\centering {\bf Secuencia normal} &  
1. El usuario accede a la pantalla de inicio de sesión.

2. El usuario introduce sus credenciales en los campos de texto y pincha en Iniciar sesión.

3. El usuario accede a la aplicación con la sesión iniciada.
\\ \hline
\centering {\bf Postcondición} &  El usuario inicia sesión en lex.gal. \\ \hline
\centering {\bf Importancia} & Vital \\ \hline
\centering {\bf Urgencia} & Inmediatamente \\ \hline
\end{tabular}
\caption{Caso de uso 01.}
\label{enlaceUC1}
\end{center}
\end{table}

\begin{table}[H]
\begin{center}
\begin{tabular}{|p{3cm}|p{10cm}|} \hline
\centering {\bf UC-02} & Cerrar sesión  \\ \hline\hline
\centering {\bf Versión} & 1.0 (05/06/2022) \\ \hline
\centering {\bf Dependencias} &  FRQ-02. \\ \hline
\centering {\bf Descripción} &  El sistema deberá comportarse tal como se describe en el siguiente caso de uso cuando el usuario trate de cerrar sesión en la aplicación. \\ \hline
\centering {\bf Precondición} &  El usuario debe de haber iniciado sesión en el sistema. \\ \hline
\centering {\bf Secuencia normal} &  
1. El usuario pincha en Cerrar sesión.

2. El usuario cierra la sesión en la aplicación.
\\ \hline
\centering {\bf Postcondición} &  El usuario cierra sesión en lex.gal. \\ \hline
\centering {\bf Importancia} & Vital \\ \hline
\centering {\bf Urgencia} & Inmediatamente \\ \hline
\end{tabular}
\caption{Caso de uso 02.}
\label{enlaceUC2}
\end{center}
\end{table}

\begin{table}[H]
\begin{center}
\begin{tabular}{|p{3cm}|p{10cm}|} \hline
\centering {\bf UC-03} & Registrar usuario  \\ \hline\hline
\centering {\bf Versión} & 1.0 (05/06/2022) \\ \hline
\centering {\bf Dependencias} &  FRQ-03. \\ \hline
\centering {\bf Descripción} &  El sistema deberá comportarse tal como se describe en el siguiente caso de uso cuando el usuario administrador trate de registrar un usuario en la aplicación. \\ \hline
\centering {\bf Precondición} &  El usuario debe de haber iniciado sesión en el sistema y debe poseer el rol de administrador. \\ \hline
\centering {\bf Secuencia normal} &  
1. El usuario administrador pincha en Registrar usuarios.

2. El usuario administrador introduce los datos del usuario a registrar y pincha en Registrar usuario.

3. El sistema almacena el usuario en la base de datos la aplicación.
\\ \hline
\centering {\bf Postcondición} &  Se registra el usuario en lex.gal. \\ \hline
\centering {\bf Importancia} & Quedaría bien \\ \hline
\centering {\bf Urgencia} & Puede esperar \\ \hline
\end{tabular}
\caption{Caso de uso 03.}
\label{enlaceUC3}
\end{center}
\end{table}

\begin{table}[H]
\begin{center}
\begin{tabular}{|p{3cm}|p{10cm}|} \hline
\centering {\bf UC-04} & Importar ley del DOG  \\ \hline\hline
\centering {\bf Versión} & 1.0 (05/06/2022) \\ \hline
\centering {\bf Dependencias} &  FRQ-04, FRQ-06. \\ \hline
\centering {\bf Descripción} &  El sistema deberá comportarse tal como se describe en el siguiente caso de uso cuando el usuario importe una ley del DOG. \\ \hline
\centering {\bf Precondición} &  El usuario debe de haber iniciado sesión en el sistema y debe tener acceso a Internet. \\ \hline
\centering {\bf Secuencia normal} &  
1. El usuario pincha en el botón de Importar del DOG.

2. El usuario introduce el texto de búsqueda junto con sus filtros y pincha en Buscar.

3. El sistema muestra las leyes encontradas en el DOG.

4. El usuario pincha en el botón de Importar de la ley deseada.

5. El sistema almacena la ley en la base de datos de la aplicación.
\\ \hline
\centering {\bf Postcondición} &  Se almacena la ley seleccionada en lex.gal. \\ \hline
\centering {\bf Excepciones} & 
1. El usuario no introduce ningún texto en el campo de búsqueda en el paso 2. A continuación, el caso de uso queda sin efecto.

2. No se encuentra ninguna ley a partir de los filtros en el paso 3. A continuación, el caso de uso queda sin efecto.
\\ \hline
\centering {\bf Importancia} & Vital \\ \hline
\centering {\bf Urgencia} & Inmediatamente \\ \hline
\end{tabular}
\caption{Caso de uso 04.}
\label{enlaceUC4}
\end{center}
\end{table}

\begin{table}[H]
\begin{center}
\begin{tabular}{|p{3cm}|p{10cm}|} \hline
\centering {\bf UC-05} & Previsualizar una ley del DOG  \\ \hline\hline
\centering {\bf Versión} & 1.0 (05/06/2022) \\ \hline
\centering {\bf Dependencias} &  FRQ-04, FRQ-05. \\ \hline
\centering {\bf Descripción} &  El sistema deberá comportarse tal como se describe en el siguiente caso de uso cuando el usuario previsualice una ley del DOG. \\ \hline
\centering {\bf Precondición} &  El usuario debe de haber iniciado sesión en el sistema y debe tener acceso a Internet. \\ \hline
\centering {\bf Secuencia normal} &  
1. El usuario pincha en el botón de Importar del DOG.

2. El usuario introduce el texto de búsqueda junto con sus filtros y pincha en Buscar.

3. El sistema muestra las leyes encontradas en el DOG.

4. El usuario pincha en el botón de Previsualizar de la ley deseada.

5. El sistema muestra al usuario el documento de la ley en la página web del DOG.
\\ \hline
\centering {\bf Postcondición} &  Se muestra al usuario la ley que desea previsualizar. \\ \hline
\centering {\bf Excepciones} & 
1. El usuario no introduce ningún texto en el campo de búsqueda en el paso 2. A continuación, el caso de uso queda sin efecto.

2. No se encuentra ninguna ley a partir de los filtros en el paso 3. A continuación, el caso de uso queda sin efecto.
\\ \hline
\centering {\bf Importancia} & Vital \\ \hline
\centering {\bf Urgencia} & Inmediatamente \\ \hline
\end{tabular}
\caption{Caso de uso 05.}
\label{enlaceUC5}
\end{center}
\end{table}

\begin{table}[H]
\begin{center}
\begin{tabular}{|p{3cm}|p{10cm}|} \hline
\centering {\bf UC-06} & Previsualizar una ley en lex.gal  \\ \hline\hline
\centering {\bf Versión} & 1.0 (05/06/2022) \\ \hline
\centering {\bf Dependencias} & FRQ-07, FRQ-08, FRQ-09. \\ \hline
\centering {\bf Descripción} &  El sistema deberá comportarse tal como se describe en el siguiente caso de uso cuando el usuario previsualice una ley de lex.gal. \\ \hline
\centering {\bf Precondición} &  El usuario debe de haber iniciado sesión en el sistema. \\ \hline
\centering {\bf Secuencia normal} &  
1. El usuario introduce el texto de búsqueda y pincha en Buscar.

2. El sistema muestra las leyes encontradas en lex.gal.

3. El usuario pincha en el botón de Previsualizar de la ley deseada.

4. El sistema muestra al usuario el documento de la ley de lex.gal.
\\ \hline
\centering {\bf Postcondición} &  Se muestra al usuario la ley que desea previsualizar. \\ \hline
\centering {\bf Excepciones} & 
1. No se encuentra ninguna ley a partir del texto en el paso 3. A continuación, el caso de uso queda sin efecto.
\\ \hline
\centering {\bf Importancia} & Vital \\ \hline
\centering {\bf Urgencia} & Inmediatamente \\ \hline
\end{tabular}
\caption{Caso de uso 06.}
\label{enlaceUC6}
\end{center}
\end{table}

\begin{table}[H]
\begin{center}
\begin{tabular}{|p{3cm}|p{10cm}|} \hline
\centering {\bf UC-07} & Validar y publicar una ley y sus leyes vinculadas en lex.gal  \\ \hline\hline
\centering {\bf Versión} & 1.0 (05/06/2022) \\ \hline
\centering {\bf Dependencias} & FRQ-07, FRQ-09, FRQ-10, FRQ-11, FRQ-12, FRQ-13, FRQ-14. \\ \hline
\centering {\bf Descripción} &  El sistema deberá comportarse tal como se describe en el siguiente caso de uso cuando el usuario trate de validar y publicar una ley y sus leyes vinculadas de lex.gal. \\ \hline
\centering {\bf Precondición} &  El usuario debe de haber iniciado sesión en el sistema. \\ \hline
\centering {\bf Secuencia normal} &  
1. El usuario propone cambios, añade anotaciones, añade leyes vinculadas o edita leyes vinculadas.

2. El usuario pincha en Validar e publicar.

3. El sistema almacena los datos de la ley y su ley vinculada, siendo estas marcadas como validadas, en la base de datos.

4. El sistema muestra un mensaje de que la ley ha sido editada.

5. El sistema redirige al usuario a la página principal de lex.gal.
\\ \hline
\centering {\bf Postcondición} &  Se almacenan los datos editados en la respectiva ley. \\ \hline
\centering {\bf Importancia} & Vital \\ \hline
\centering {\bf Urgencia} & Inmediatamente \\ \hline
\end{tabular}
\caption{Caso de uso 07.}
\label{enlaceUC7}
\end{center}
\end{table}

\begin{table}[H]
\begin{center}
\begin{tabular}{|p{3cm}|p{10cm}|} \hline
\centering {\bf UC-08} & Guardar como borrador una ley y sus leyes vinculadas en lex.gal  \\ \hline\hline
\centering {\bf Versión} & 1.0 (05/06/2022) \\ \hline
\centering {\bf Dependencias} & FRQ-07, FRQ-09, FRQ-10, FRQ-11, FRQ-12, FRQ-13, FRQ-14. \\ \hline
\centering {\bf Descripción} &  El sistema deberá comportarse tal como se describe en el siguiente caso de uso cuando el usuario trate de guardar como borrador una ley y sus leyes vinculadas de lex.gal. \\ \hline
\centering {\bf Precondición} &  El usuario debe de haber iniciado sesión en el sistema. \\ \hline
\centering {\bf Secuencia normal} &  
1. El usuario propone cambios, añade anotaciones, añade leyes vinculadas o edita leyes vinculadas.

2. El usuario pincha en Guardar como borrador.

3. El sistema almacena los datos de la ley y su ley vinculada, siendo estas marcadas como borrador, en la base de datos.

4. El sistema muestra un mensaje de que la ley ha sido editada.

5. El sistema redirige al usuario a la página principal de lex.gal.
\\ \hline
\centering {\bf Postcondición} &  Se almacenan los datos editados en la respectiva ley. \\ \hline
\centering {\bf Importancia} & Vital \\ \hline
\centering {\bf Urgencia} & Inmediatamente \\ \hline
\end{tabular}
\caption{Caso de uso 08.}
\label{enlaceUC8}
\end{center}
\end{table}

\begin{table}[H]
\begin{center}
\begin{tabular}{|p{3cm}|p{10cm}|} \hline
\centering {\bf UC-09} & Eliminar una ley en lex.gal  \\ \hline\hline
\centering {\bf Versión} & 1.0 (05/06/2022) \\ \hline
\centering {\bf Dependencias} &  FRQ-15. \\ \hline
\centering {\bf Descripción} &  El sistema deberá comportarse tal como se describe en el siguiente caso de uso cuando el usuario elimine una ley de lex.gal. \\ \hline
\centering {\bf Precondición} &  El usuario debe de haber iniciado sesión en el sistema. \\ \hline
\centering {\bf Secuencia normal} &  
1. El usuario pincha en Eliminar lei de lex.gal.

2. El sistema elimina la ley de lex.gal.

3. El sistema muestra un mensaje de que la ley ha sido eliminada.

4. El sistema redirige al usuario a la página principal de lex.gal.
\\ \hline
\centering {\bf Postcondición} &  La ley es eliminada de lex.gal. \\ \hline
\centering {\bf Importancia} & Quedaría bien \\ \hline
\centering {\bf Urgencia} & Puede esperar \\ \hline
\end{tabular}
\caption{Caso de uso 09.}
\label{enlaceUC9}
\end{center}
\end{table}