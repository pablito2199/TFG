\section{Requisitos no funcionales}

En cuanto a los requisitos no funcionales \cite{requisitos}, estos son los requisitos que especifican criterios que pueden usarse para juzgar la operación de un sistema, es decir, las características de funcionamiento.
\\

Podemos identificar como requisitos funcionales aquellos requisitos que afectan al rendimiento, seguridad, restricciones, etc.

\begin{table}[H]
\begin{center}
\begin{tabular}{|p{3cm}|p{10cm}|} \hline
\centering {\bf NFR-01} & Tiempo de respuesta $<$ 2 segundos  \\ \hline\hline
\centering {\bf Versión} & 1.0 (05/06/2022) \\ \hline
\centering {\bf Descripción} & El sistema deberá responder en menos de 2 segundos, sin tener en cuenta el tiempo que tarde la red en enviar la información de vuelta al usuario. \\ \hline
\centering {\bf Importancia} & Vital \\ \hline
\centering {\bf Urgencia} & Inmediatamente \\ \hline
\end{tabular}
\caption{Requisito no funcional 01.}
\label{enlaceNFR1}
\end{center}
\end{table}

\begin{table}[H]
\begin{center}
\begin{tabular}{|p{3cm}|p{10cm}|} \hline
\centering {\bf NFR-02} & Los datos sensibles del usuario se almacenan encriptados  \\ \hline\hline
\centering {\bf Versión} & 1.0 (05/06/2022) \\ \hline
\centering {\bf Descripción} & El sistema deberá encriptar los datos sensibles del usuario como las contraseñas en la base de datos de lex.gal. Se utilizará un algoritmo SHA-256 para encriptar. \\ \hline
\centering {\bf Importancia} & Vital \\ \hline
\centering {\bf Urgencia} & Inmediatamente \\ \hline
\end{tabular}
\caption{Requisito no funcional 02.}
\label{enlaceNFR2}
\end{center}
\end{table}

\begin{table}[H]
\begin{center}
\begin{tabular}{|p{3cm}|p{10cm}|} \hline
\centering {\bf NFR-03} & No permitir que un rol realice operaciones de otro  \\ \hline\hline
\centering {\bf Versión} & 1.0 (05/06/2022) \\ \hline
\centering {\bf Descripción} & El sistema deberá restringir aquellas operaciones exclusivas del administrador para el resto de usuarios. \\ \hline
\centering {\bf Importancia} & Vital \\ \hline
\centering {\bf Urgencia} & Inmediatamente \\ \hline
\end{tabular}
\caption{Requisito no funcional 03.}
\label{enlaceNFR3}
\end{center}
\end{table}

\begin{table}[H]
\begin{center}
\begin{tabular}{|p{3cm}|p{10cm}|} \hline
\centering {\bf NFR-04} & Permanecer operativo durante el horario de trabajo \\ \hline\hline
\centering {\bf Versión} & 1.0 (05/06/2022) \\ \hline
\centering {\bf Descripción} & El sistema deberá permanecer operativo durante todo el horario de trabajo de los empleados de lex.gal.  \\ \hline
\centering {\bf Importancia} & Vital \\ \hline
\centering {\bf Urgencia} & Inmediatamente \\ \hline
\end{tabular}
\caption{Requisito no funcional 04.}
\label{enlaceNFR4}
\end{center}
\end{table}

\begin{table}[H]
\begin{center}
\begin{tabular}{|p{3cm}|p{10cm}|} \hline
\centering {\bf NFR-05} & Tratamiento de documentos en lenguajes de etiquetas  \\ \hline\hline
\centering {\bf Versión} & 1.0 (05/06/2022) \\ \hline
\centering {\bf Descripción} & El sistema deberá restringir el tratamiento de documentos a documentos escritos en lenguajes de etiquetas, como HTML o XML. \\ \hline
\centering {\bf Importancia} & Vital \\ \hline
\centering {\bf Urgencia} & Inmediatamente \\ \hline
\end{tabular}
\caption{Requisito no funcional 05.}
\label{enlaceNFR5}
\end{center}
\end{table}

\begin{table}[H]
\begin{center}
\begin{tabular}{|p{3cm}|p{10cm}|} \hline
\centering {\bf NFR-06} & Tiempo de aprendizaje por el usuario $<$ 1h  \\ \hline\hline
\centering {\bf Versión} & 1.0 (05/06/2022) \\ \hline
\centering {\bf Descripción} & El sistema deberá tener un proceso de aprendizaje rápido y sencillo para el usuario, que en todo caso deberá ser menor de 1 hora. \\ \hline
\centering {\bf Importancia} & Vital \\ \hline
\centering {\bf Urgencia} & Inmediatamente \\ \hline
\end{tabular}
\caption{Requisito no funcional 06.}
\label{enlaceNFR6}
\end{center}
\end{table}

\begin{table}[H]
\begin{center}
\begin{tabular}{|p{3cm}|p{10cm}|} \hline
\centering {\bf NFR-07} & Proporcionar mensajes de apoyo al usuario  \\ \hline\hline
\centering {\bf Versión} & 1.0 (05/06/2022) \\ \hline
\centering {\bf Descripción} & El sistema deberá proporcionar mensajes de apoyo al usuario en todas las operaciones, con el objetivo de facilitar la navegación en la aplicación. \\ \hline
\centering {\bf Importancia} & Quedaría bien \\ \hline
\centering {\bf Urgencia} & Puede esperar \\ \hline
\end{tabular}
\caption{Requisito no funcional 07.}
\label{enlaceNFR7}
\end{center}
\end{table}

\begin{table}[H]
\begin{center}
\begin{tabular}{|p{3cm}|p{10cm}|} \hline
\centering {\bf NFR-08} & Uso de tecnologías web y servicios RESTful  \\ \hline\hline
\centering {\bf Versión} & 1.0 (05/06/2022) \\ \hline
\centering {\bf Descripción} & El sistema deberá emplear tecnologías web para realizar la aplicación, incluyendo HTML5, Tailwind \cite{tailwind} como framework de CSS y React.js \cite{react} como framework de JavaScript. Además, todos los servicios serán RESTful, y serán implementados con Spring Boot \cite{spring}. \\ \hline
\centering {\bf Importancia} & Vital \\ \hline
\centering {\bf Urgencia} & Inmediatamente \\ \hline
\end{tabular}
\caption{Requisito no funcional 08.}
\label{enlaceNFR8}
\end{center}
\end{table}

\begin{table}[H]
\begin{center}
\begin{tabular}{|p{3cm}|p{10cm}|} \hline
\centering {\bf NFR-09} & Base de datos NoSQL  \\ \hline\hline
\centering {\bf Versión} & 1.0 (05/06/2022) \\ \hline
\centering {\bf Descripción} & El sistema deberá almacenar todos los datos en una base de datos NoSQL de MongoDB \cite{mongodb} \\ \hline
\centering {\bf Importancia} & Vital \\ \hline
\centering {\bf Urgencia} & Inmediatamente \\ \hline
\end{tabular}
\caption{Requisito no funcional 09.}
\label{enlaceNFR9}
\end{center}
\end{table}

\begin{table}[H]
\begin{center}
\begin{tabular}{|p{3cm}|p{10cm}|} \hline
\centering {\bf NFR-10} & Generación de tokens JWT  \\ \hline\hline
\centering {\bf Versión} & 1.0 (05/06/2022) \\ \hline
\centering {\bf Descripción} & El sistema deberá generar tokens JWT \cite{jwt} cada vez que un usuario inicie sesión en la aplicación. De esta forma, los datos del usuario no quedan tan accesibles en la web, y será necesario utilizar un algoritmo hash para desencriptarlos. En todo caso la contraseña no será almacenada en el token JWT. \\ \hline
\centering {\bf Importancia} & Vital \\ \hline
\centering {\bf Urgencia} & Inmediatamente \\ \hline
\end{tabular}
\caption{Requisito no funcional 10.}
\label{enlaceNFR10}
\end{center}
\end{table}