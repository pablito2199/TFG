\section{Requisitos de información}

Por último, los requisitos de información \cite{requisitos}, son aquellos requisitos que indican algún tipo de información, la cual es necesaria para el correcto funcionamiento del sistema.
\\

En el caso de esta aplicación, se han localizado dos:

\begin{table}[H]
\begin{center}
\begin{tabular}{|p{3cm}|p{10cm}|} \hline
\centering {\bf IRQ-01} & Usuario  \\ \hline\hline
\centering {\bf Versión} & 1.0 (05/06/2022) \\ \hline
\centering {\bf Descripción} & El sistema deberá almacenar la información correspondiente a los usuarios del sistema. \\ \hline
\centering {\bf Datos específicos} & 
- Correo electrónico

- Nombre

- Apellidos

- Contraseña

- Roles del usuario
\\ \hline
\centering {\bf Importancia} & Vital \\ \hline
\centering {\bf Urgencia} & Inmediatamente \\ \hline
\end{tabular}
\caption{Requisito de información 01.}
\label{enlaceIRQ1}
\end{center}
\end{table}

\begin{table}[H]
\begin{center}
\begin{tabular}{|p{3cm}|p{10cm}|} \hline
\centering {\bf IRQ-02} & Documento  \\ \hline\hline
\centering {\bf Versión} & 1.0 (05/06/2022) \\ \hline
\centering {\bf Descripción} & El sistema deberá almacenar la información correspondiente a las leyes de lex.gal. \\ \hline
\centering {\bf Datos específicos}  & 
- Id

- Sumario

- Estado de la ley (validada/borrador)

- Documento original

- Documento con los cambios realizados

- Datos de cabecera

- Cambios propuestos sobre la propia ley

- Anotaciones realizadas

- Leyes vinculadas

- Cambios propuestos sobre leyes vinculadas

- Roles del usuario
. \\ \hline
\centering {\bf Importancia} & Vital \\ \hline
\centering {\bf Urgencia} & Inmediatamente \\ \hline
\end{tabular}
\caption{Requisito de información 02.}
\label{enlaceIRQ2}
\end{center}
\end{table}