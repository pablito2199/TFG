\section{Requisitos funcionales}

Los requisitos funcionales \cite{requisitos} son aquellos que definen una funcionalidad del sistema de software o sus componentes, tanto para las entradas de datos como para el comportamiento en situaciones específicas.
\\

En el caso de esta aplicación se han localizado dos grupos de requisitos funcionales: los asociados con operaciones de usuarios, y los que se relacionan con operaciones asociadas a leyes.
\\

En cuanto a requisitos establecidos para usuarios, encontramos tres requisitos funcionales: inicio de sesión, cerrar sesión y registrar usuarios. Solamente es de vital importancia el requisito de iniciar sesión, pues es necesario para realizar operaciones en el sistema.
\\

Por parte de los requisitos asociados a leyes, encontramos requisitos relacionados con búsquedas, importación, edición, previsualización y borrado. Los requisitos más importantes en este caso son los relacionados con la edición de un documento, ya que es la base de la herramienta realizada en este TFG.
\\

Los requisitos funcionales pueden ser consultados en el \hyperref[APRequisitosFuncionales]{Apéndice C.1. Requisitos funcionales}.