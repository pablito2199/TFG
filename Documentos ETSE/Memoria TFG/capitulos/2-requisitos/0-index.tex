\chapter{Especificación de requisitos}
\label{enlaceespecificacion}

Para proceder a especificar los requisitos se ha realizado un profundo análisis de los objetivos de la aplicación. Estos requisitos se han buscado teniendo en cuenta aspectos funcionales, de rendimiento, de seguridad, de información, etc.
\\

Las plantillas empleadas para las tablas de los requisitos y casos de uso se encuentran en los apéndices: la \hyperref[enlaceFRQX]{D.1. Plantilla de requisitos funcionales}, la \hyperref[enlaceNFRX]{D.2. Plantilla de requisitos no funcionales}, la \hyperref[enlaceIRQX]{D.3. Plantilla de requisitos de información} y la \hyperref[enlaceUCX]{D.4. Plantilla de casos de uso}.
Estas plantillas están inspiradas en las tablas empleadas en el REM \cite{rem} para la gestión de requisitos.
\\

La importancia de los requisitos será medida según el grado de la necesidad del requisito en la aplicación, siendo en el caso de este proyecto ``Vital" para los requisitos de mayor importancia, y, ``Quedaría bien" para los menos importantes. Por su parte, la urgencia, indicará aquellas partes que se han de implementar con mayor rapidez, siendo ``Inmediatamente" las que tienen una mayor prioridad, y, ``Puede esperar", para los requisitos que no son estrictamente necesarios.

\section{Requisitos funcionales}
\label{APRequisitosFuncionales}

\begin{table}[H]
\begin{center}
\begin{tabular}{|p{3cm}|p{10cm}|} \hline
\centering {\bf FRQ-01} & Iniciar sesión  \\ \hline\hline
\centering {\bf Versión} & 1.0 (05/06/2022) \\ \hline
\centering {\bf Descripción} & El sistema deberá permitir al usuario iniciar sesión en la aplicación. Para ello ha de introducir su email y su contraseña, generando un token JWT que permite el uso de la aplicación de forma segura. \\ \hline
\centering {\bf Importancia} & Vital \\ \hline
\centering {\bf Urgencia} & Inmediatamente \\ \hline
\end{tabular}
\caption{Requisito funcional 01. Iniciar sesión.}
\label{enlaceFRQ1}
\end{center}
\end{table}

\begin{table}[H]
\begin{center}
\begin{tabular}{|p{3cm}|p{10cm}|} \hline
\centering {\bf FRQ-02} & Cerrar sesión  \\ \hline\hline
\centering {\bf Versión} & 1.0 (05/06/2022) \\ \hline
\centering {\bf Descripción} & El sistema deberá permitir al usuario cerrar sesión en la aplicación. En este caso se elimina el token JWT del almacenamiento local, cerrando la sesión. \\ \hline
\centering {\bf Importancia} & Quedaría bien \\ \hline
\centering {\bf Urgencia} & Puede esperar \\ \hline
\end{tabular}
\caption{Requisito funcional 02. Cerrar sesión.}
\label{enlaceFRQ2}
\end{center}
\end{table}

\begin{table}[H]
\begin{center}
\begin{tabular}{|p{3cm}|p{10cm}|} \hline
\centering {\bf FRQ-03} & Registrar usuarios  \\ \hline\hline
\centering {\bf Versión} & 1.0 (05/06/2022) \\ \hline
\centering {\bf Descripción} & El sistema deberá permitir al usuario con rol de administrador registrar usuarios. Para ello, el administrador introducirá el email, nombre, apellidos, contraseña y rol del usuario a registrar. \\ \hline
\centering {\bf Importancia} & Quedaría bien \\ \hline
\centering {\bf Urgencia} & Puede esperar \\ \hline
\end{tabular}
\caption{Requisito funcional 03. Registrar usuarios.}
\label{enlaceFRQ3}
\end{center}
\end{table}

\begin{table}[H]
\begin{center}
\begin{tabular}{|p{3cm}|p{10cm}|} \hline
\centering {\bf FRQ-04} & Búsqueda de leyes en el DOG  \\ \hline\hline
\centering {\bf Versión} & 1.0 (05/06/2022) \\ \hline
\centering {\bf Descripción} & El sistema deberá permitir al usuario buscar las leyes deseadas en el DOG. Se podrán filtrar leyes según los siguientes parámetros de búsqueda: texto, buscar solo en el título, buscar por frase exacta, número del DOG desde/hasta, criterio de ordenación, colectivo, organización, rango, sección y área temática. \\ \hline
\centering {\bf Importancia} & Vital \\ \hline
\centering {\bf Urgencia} & Inmediatamente \\ \hline
\end{tabular}
\caption{Requisito funcional 04. Búsqueda de leyes en el DOG.}
\label{enlaceFRQ4}
\end{center}
\end{table}

\begin{table}[H]
\begin{center}
\begin{tabular}{|p{3cm}|p{10cm}|} \hline
\centering {\bf FRQ-05} & Previsualizar leyes en el DOG  \\ \hline\hline
\centering {\bf Versión} & 1.0 (05/06/2022) \\ \hline
\centering {\bf Descripción} & El sistema permitirá al usuario previsualizar una ley del DOG en su propia página web. Se abrirá una pestaña nueva del DOG con la ley seleccionada. \\ \hline
\centering {\bf Importancia} & Quedaría bien \\ \hline
\centering {\bf Urgencia} & Puede esperar \\ \hline
\end{tabular}
\caption{Requisito funcional 05. Previsualizar leyes en el DOG.}
\label{enlaceFRQ5}
\end{center}
\end{table}

\begin{table}[H]
\begin{center}
\begin{tabular}{|p{3cm}|p{10cm}|} \hline
\centering {\bf FRQ-06} & Importar leyes del DOG  \\ \hline\hline
\centering {\bf Versión} & 1.0 (05/06/2022) \\ \hline
\centering {\bf Descripción} & El sistema deberá permitir al usuario importar una ley del DOG a la base de datos de lex.gal. Importará todos los datos relevantes de una ley: sumario, publicador, fecha de publicación en el DOG, colectivo, organización, rango, sección, área temática, número del DOG y el texto del documento. Todas las leyes serán importadas como borrador. \\ \hline
\centering {\bf Importancia} & Vital \\ \hline
\centering {\bf Urgencia} & Inmediatamente \\ \hline
\end{tabular}
\caption{Requisito funcional 06. Importar leyes del DOG.}
\label{enlaceFRQ6}
\end{center}
\end{table}

\begin{table}[H]
\begin{center}
\begin{tabular}{|p{3cm}|p{10cm}|} \hline
\centering {\bf FRQ-07} & Búsqueda de leyes en lex.gal  \\ \hline\hline
\centering {\bf Versión} & 1.0 (05/06/2022) \\ \hline
\centering {\bf Descripción} & El sistema deberá permitir al usuario buscar las leyes en la base de datos de lex.gal. En este caso solo se podrá filtrar por el texto contenido en el sumario. \\ \hline
\centering {\bf Importancia} & Vital \\ \hline
\centering {\bf Urgencia} & Inmediatamente \\ \hline
\end{tabular}
\caption{Requisito funcional 07. Búsqueda de leyes en lex.gal.}
\label{enlaceFRQ7}
\end{center}
\end{table}

\begin{table}[H]
\begin{center}
\begin{tabular}{|p{3cm}|p{10cm}|} \hline
\centering {\bf FRQ-08} & Previsualizar leyes en lex.gal  \\ \hline\hline
\centering {\bf Versión} & 1.0 (05/06/2022) \\ \hline
\centering {\bf Descripción} & El sistema deberá permitir al usuario previsualizar el contenido del documento de una ley almacenada en lex.gal. \\ \hline
\centering {\bf Importancia} & Vital \\ \hline
\centering {\bf Urgencia} & Inmediatamente \\ \hline
\end{tabular}
\caption{Requisito funcional 08. Previsualizar leyes en lex.gal.}
\label{enlaceFRQ8}
\end{center}
\end{table}

\begin{table}[H]
\begin{center}
\begin{tabular}{|p{3cm}|p{10cm}|} \hline
\centering {\bf FRQ-09} & Cargar leyes de lex.gal  \\ \hline\hline
\centering {\bf Versión} & 1.0 (05/06/2022) \\ \hline
\centering {\bf Descripción} & El sistema deberá permitir al usuario cargar los datos de cualquier ley de lex.gal. Se importarán los datos solicitados en todo momento, como el sumario, texto del documento, datos de cabecera, leyes vinculadas, cambios realizados y las anotaciones añadidas. \\ \hline
\centering {\bf Importancia} & Vital \\ \hline
\centering {\bf Urgencia} & Inmediatamente \\ \hline
\end{tabular}
\caption{Requisito funcional 09. Cargar leyes de lex.gal.}
\label{enlaceFRQ9}
\end{center}
\end{table}

\begin{table}[H]
\begin{center}
\begin{tabular}{|p{3cm}|p{10cm}|} \hline
\centering {\bf FRQ-10} & Proponer cambios en una ley de lex.gal  \\ \hline\hline
\centering {\bf Versión} & 1.0 (05/06/2022) \\ \hline
\centering {\bf Descripción} & El sistema deberá permitir al usuario proponer cambios sobre cualquier párrafo de una ley de lex.gal. Los cambios serán introducidos en el documento para poder previsualizar la norma con los cambios propuestos. \\ \hline
\centering {\bf Importancia} & Quedaría bien \\ \hline
\centering {\bf Urgencia} & Puede esperar \\ \hline
\end{tabular}
\caption{Requisito funcional 10. Proponer cambios en una ley de lex.gal.}
\label{enlaceFRQ10}
\end{center}
\end{table}

\begin{table}[H]
\begin{center}
\begin{tabular}{|p{3cm}|p{10cm}|} \hline
\centering {\bf FRQ-11} & Añadir anotaciones sobre párrafos de una ley de lex.gal  \\ \hline\hline
\centering {\bf Versión} & 1.0 (05/06/2022) \\ \hline
\centering {\bf Descripción} & El sistema deberá permitir al usuario añadir anotaciones sobre cualquier párrafo del documento de una ley de lex.gal. Se podrán añadir comentarios sobre la anotación realizada. \\ \hline
\centering {\bf Importancia} & Quedaría bien \\ \hline
\centering {\bf Urgencia} & Puede esperar \\ \hline
\end{tabular}
\caption{Requisito funcional 11. Añadir anotaciones sobre párrafos de una ley de lex.gal.}
\label{enlaceFRQ11}
\end{center}
\end{table}

\begin{table}[H]
\begin{center}
\begin{tabular}{|p{3cm}|p{10cm}|} \hline
\centering {\bf FRQ-12} & Cargar leyes vinculadas a una ley de lex.gal  \\ \hline\hline
\centering {\bf Versión} & 1.0 (05/06/2022) \\ \hline
\centering {\bf Descripción} & El sistema deberá permitir al usuario cargar las leyes vinculadas a una ley de lex.gal. Se importará tanto el nombre de una ley vinculada, como el documento en el caso de que se realice alguna modificación sobre ella. \\ \hline
\centering {\bf Importancia} & Vital \\ \hline
\centering {\bf Urgencia} & Inmediatamente \\ \hline
\end{tabular}
\caption{Requisito funcional 12. Cargar leyes vinculadas a una ley de lex.gal.}
\label{enlaceFRQ12}
\end{center}
\end{table}

\begin{table}[H]
\begin{center}
\begin{tabular}{|p{3cm}|p{10cm}|} \hline
\centering {\bf FRQ-13} & Añadir leyes vinculadas a una ley de lex.gal  \\ \hline\hline
\centering {\bf Versión} & 1.0 (05/06/2022) \\ \hline
\centering {\bf Descripción} & El sistema deberá permitir al usuario añadir manualmente una ley vinculada a cualquier ley de lex.gal. Se mostrará en la lista de leyes vinculadas a la propia ley. \\ \hline
\centering {\bf Importancia} & Vital \\ \hline
\centering {\bf Urgencia} & Inmediatamente \\ \hline
\end{tabular}
\caption{Requisito funcional 13. Añadir leyes vinculadas a una ley de lex.gal.}
\label{enlaceFRQ13}
\end{center}
\end{table}

\begin{table}[H]
\begin{center}
\begin{tabular}{|p{3cm}|p{10cm}|} \hline
\centering {\bf FRQ-14} & Proponer cambios en leyes vinculadas a una ley de lex.gal  \\ \hline\hline
\centering {\bf Versión} & 1.0 (05/06/2022) \\ \hline
\centering {\bf Descripción} & El sistema deberá permitir al usuario proponer cambios sobre cualquier ley vinculada que modifique una ley de lex.gal. Se mostrará en la ley principal donde se han de realizar los cambios en la ley vinculada, permitiendo un método sencillo para el usuario a la hora de proponer el cambio.  \\ \hline
\centering {\bf Importancia} & Vital \\ \hline
\centering {\bf Urgencia} & Inmediatamente \\ \hline
\end{tabular}
\caption{Requisito funcional 14. Proponer cambios en leyes vinculadas a una ley de lex.gal.}
\label{enlaceFRQ14}
\end{center}
\end{table}

\begin{table}[H]
\begin{center}
\begin{tabular}{|p{3cm}|p{10cm}|} \hline
\centering {\bf FRQ-15} & Eliminar una ley de lex.gal  \\ \hline\hline
\centering {\bf Versión} & 1.0 (05/06/2022) \\ \hline
\centering {\bf Descripción} & El sistema deberá permitir al usuario eliminar cualquier ley de lex.gal. La ley será eliminada definitivamente de la base de datos. \\ \hline
\centering {\bf Importancia} & Quedaría bien \\ \hline
\centering {\bf Urgencia} & Puede esperar \\ \hline
\end{tabular}
\caption{Requisito funcional 15. Eliminar una ley de lex.gal.}
\label{enlaceFRQ15}
\end{center}
\end{table}
\section{Requisitos no funcionales}

En cuanto a los requisitos no funcionales \cite{requisitos}, estos son los requisitos que especifican criterios que pueden usarse para juzgar la operación de un sistema, es decir, las características de funcionamiento.
\\

Podemos identificar como requisitos no funcionales aquellos requisitos que afectan al rendimiento, seguridad, restricciones, etc.
\\

Los requisitos no funcionales pueden ser consultados en el \hyperref[APRequisitosNoFuncionales]{Apéndice C.2. Requisitos no funcionales}.
\section{Requisitos de información}
\label{APRequisitosInformacion}

\begin{table}[H]
\begin{center}
\begin{tabular}{|p{3cm}|p{10cm}|} \hline
\centering {\bf IRQ-01} & Usuario  \\ \hline\hline
\centering {\bf Versión} & 1.0 (05/06/2022) \\ \hline
\centering {\bf Descripción} & El sistema deberá almacenar la información correspondiente a los usuarios del sistema. \\ \hline
\centering {\bf Datos específicos} & 
- Correo electrónico

- Nombre

- Apellidos

- Contraseña

- Roles del usuario
\\ \hline
\centering {\bf Importancia} & Vital \\ \hline
\centering {\bf Urgencia} & Inmediatamente \\ \hline
\end{tabular}
\caption{Requisito de información 01.}
\label{enlaceIRQ1}
\end{center}
\end{table}

\begin{table}[H]
\begin{center}
\begin{tabular}{|p{3cm}|p{10cm}|} \hline
\centering {\bf IRQ-02} & Documento  \\ \hline\hline
\centering {\bf Versión} & 1.0 (05/06/2022) \\ \hline
\centering {\bf Descripción} & El sistema deberá almacenar la información correspondiente a las leyes de lex.gal. \\ \hline
\centering {\bf Datos específicos}  & 
- Id

- Sumario

- Estado de la ley (validada/borrador)

- Documento original

- Documento con los cambios realizados

- Datos de cabecera

- Cambios propuestos sobre la propia ley

- Anotaciones realizadas

- Leyes vinculadas

- Cambios propuestos sobre leyes vinculadas
\\ \hline
\centering {\bf Importancia} & Vital \\ \hline
\centering {\bf Urgencia} & Inmediatamente \\ \hline
\end{tabular}
\caption{Requisito de información 02.}
\label{enlaceIRQ2}
\end{center}
\end{table}
\section{Casos de uso}

Un caso de uso \cite{casodeuso} es cualquier acción o actividad que se realiza en el sistema. A continuación se presentan los diferentes casos de uso identificados entre las funcionalidades de la aplicación.
\\

Los casos de uso pueden ser consultados en el \hyperref[APCasosUso]{Apéndice C.4. Casos de uso}.