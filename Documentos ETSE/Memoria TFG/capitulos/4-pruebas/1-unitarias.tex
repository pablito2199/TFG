\section{Pruebas unitarias}

Las pruebas unitarias \cite{pruebaunitaria} tienen como objetivo comprobar que las funcionalidades de la aplicación cumplen los objetivos esperados a la hora de introducir parámetros válidos e inválidos en la aplicación. Es por ello que estas pruebas han de ser siempre lo más completas posible, y deben abarcar un gran número de casos de entrada.
\\

Para especificar cada caso de prueba unitaria se ha empleado la \hyperref[enlacePUX]{D.5. Plantilla de pruebas unitarias}. Cabe mencionar también que en las dependencias se incluyen los requisitos funcionales especificados en el \hyperref[APRequisitosFuncionales]{Apéndice C.1. Requisitos funcionales}. Los distintos casos se muestran a continuación.

\begin{table}[H]
\begin{center}
\begin{tabular}{|p{3cm}|p{10cm}|} \hline
\centering {\bf PU-01} & Inicio de sesión con parámetros correctos  \\ \hline\hline
\centering {\bf Versión} & 1.0 (08/06/2022) \\ \hline
\centering {\bf Dependencias} &  FRQ-01. \\ \hline
\centering {\bf Descripción} &  El usuario introduce su email y contraseña e inicia sesión. \\ \hline
\centering {\bf Criterio de aceptación} & La sesión se inicia correctamente. \\ \hline
\centering {\bf Estado} & Superada \\ \hline
\end{tabular}
\caption{Prueba unitaria 01. Inicio de sesión con parámetros correctos.}
\label{enlacePU1}
\end{center}
\end{table}

\begin{table}[H]
\begin{center}
\begin{tabular}{|p{3cm}|p{10cm}|} \hline
\centering {\bf PU-02} & Inicio de sesión con parámetros incorrectos  \\ \hline\hline
\centering {\bf Versión} & 1.0 (08/06/2022) \\ \hline
\centering {\bf Dependencias} & FRQ-01. \\ \hline
\centering {\bf Descripción} &  El usuario introduce un email y/o contraseña incorrectos e inicia sesión. \\ \hline
\centering {\bf Criterio de aceptación} & Se muestra un mensaje de error al usuario y no se inicia la sesión. \\ \hline
\centering {\bf Estado} & Superada \\ \hline
\end{tabular}
\caption{Prueba unitaria 02. Inicio de sesión con parámetros incorrectos.}
\label{enlacePU2}
\end{center}
\end{table}

\begin{table}[H]
\begin{center}
\begin{tabular}{|p{3cm}|p{10cm}|} \hline
\centering {\bf PU-03} & Cierre de sesión  \\ \hline\hline
\centering {\bf Versión} & 1.0 (08/06/2022) \\ \hline
\centering {\bf Dependencias} & FRQ-02. \\ \hline
\centering {\bf Descripción} &  El usuario pincha en el botón de cerrar sesión. \\ \hline
\centering {\bf Criterio de aceptación} & Se cierra la sesión del usuario en la aplicación. \\ \hline
\centering {\bf Estado} & Superada \\ \hline
\end{tabular}
\caption{Prueba unitaria 03. Cierre de sesión.}
\label{enlacePU3}
\end{center}
\end{table}

\begin{table}[H]
\begin{center}
\begin{tabular}{|p{3cm}|p{10cm}|} \hline
\centering {\bf PU-04} & Registro de usuario con parámetros válidos  \\ \hline\hline
\centering {\bf Versión} & 1.0 (08/06/2022) \\ \hline
\centering {\bf Dependencias} & FRQ-03. \\ \hline
\centering {\bf Descripción} &  El usuario administrador registra un usuario con todos los parámetros de registro válidos. \\ \hline
\centering {\bf Criterio de aceptación} & Se registra al usuario en la aplicación. \\ \hline
\centering {\bf Estado} & Superada \\ \hline
\end{tabular}
\caption{Prueba unitaria 04. Registro de usuario con parámetros válidos.}
\label{enlacePU4}
\end{center}
\end{table}

\begin{table}[H]
\begin{center}
\begin{tabular}{|p{3cm}|p{10cm}|} \hline
\centering {\bf PU-05} & Registro de usuario con parámetros inválidos  \\ \hline\hline
\centering {\bf Versión} & 1.0 (08/06/2022) \\ \hline
\centering {\bf Dependencias} & FRQ-03. \\ \hline
\centering {\bf Descripción} &  El usuario administrador intenta registrar un usuario con uno de sus parámetros con condiciones no válidas para el registro. \\ \hline
\centering {\bf Criterio de aceptación} & Se muestra un mensaje de error al usuario administrador indicando cuál es el campo inválido. \\ \hline
\centering {\bf Estado} & Superada \\ \hline
\end{tabular}
\caption{Prueba unitaria 05. Registro de usuario con parámetros inválidos.}
\label{enlacePU5}
\end{center}
\end{table}

\begin{table}[H]
\begin{center}
\begin{tabular}{|p{3cm}|p{10cm}|} \hline
\centering {\bf PU-06} & Búsqueda de leyes en el DOG con filtros válidos  \\ \hline\hline
\centering {\bf Versión} & 1.0 (08/06/2022) \\ \hline
\centering {\bf Dependencias} & FRQ-04. \\ \hline
\centering {\bf Descripción} &  El usuario busca cualquier ley en el DOG con filtros válidos. \\ \hline
\centering {\bf Criterio de aceptación} & Se muestran al usuario las leyes encontradas en el DOG. \\ \hline
\centering {\bf Estado} & Superada \\ \hline
\end{tabular}
\caption{Prueba unitaria 06. Búsqueda de leyes en el DOG con filtros válidos.}
\label{enlacePU6}
\end{center}
\end{table}

\begin{table}[H]
\begin{center}
\begin{tabular}{|p{3cm}|p{10cm}|} \hline
\centering {\bf PU-07} & Búsqueda de leyes en el DOG sin introducir texto  \\ \hline\hline
\centering {\bf Versión} & 1.0 (08/06/2022) \\ \hline
\centering {\bf Dependencias} & FRQ-04. \\ \hline
\centering {\bf Descripción} &  El usuario busca cualquier ley en el DOG sin introducir nada en el campo de texto de búsqueda. \\ \hline
\centering {\bf Criterio de aceptación} & Se muestra al usuario un mensaje indicando que es necesario introducir un texto para buscar. \\ \hline
\centering {\bf Estado} & Superada \\ \hline
\end{tabular}
\caption{Prueba unitaria 07. Búsqueda de leyes en el DOG sin introducir texto.}
\label{enlacePU7}
\end{center}
\end{table}

\begin{table}[H]
\begin{center}
\begin{tabular}{|p{3cm}|p{10cm}|} \hline
\centering {\bf PU-08} & Previsualizar una ley en el DOG  \\ \hline\hline
\centering {\bf Versión} & 1.0 (08/06/2022) \\ \hline
\centering {\bf Dependencias} & FRQ-05. \\ \hline
\centering {\bf Descripción} &  El usuario intenta previsualizar una ley en el DOG pinchando en el icono del ojo de las acciones de una ley del DOG. \\ \hline
\centering {\bf Criterio de aceptación} & Se muestra al usuario la ley en la página del DOG. \\ \hline
\centering {\bf Estado} & Superada \\ \hline
\end{tabular}
\caption{Prueba unitaria 08. Previsualizar una ley en el DOG.}
\label{enlacePU8}
\end{center}
\end{table}

\begin{table}[H]
\begin{center}
\begin{tabular}{|p{3cm}|p{10cm}|} \hline
\centering {\bf PU-09} & Importar ley del DOG en lex.gal  \\ \hline\hline
\centering {\bf Versión} & 1.0 (08/06/2022) \\ \hline
\centering {\bf Dependencias} & FRQ-06. \\ \hline
\centering {\bf Descripción} &  El usuario intenta importar una ley del DOG en lex.gal haciendo click en el botón de importar. \\ \hline
\centering {\bf Criterio de aceptación} & La ley se almacena en lex.gal y se muestra al usuario un mensaje de que se ha almacenado correctamente. \\ \hline
\centering {\bf Estado} & Superada \\ \hline
\end{tabular}
\caption{Prueba unitaria 09. Importar ley del DOG en lex.gal.}
\label{enlacePU9}
\end{center}
\end{table}

\begin{table}[H]
\begin{center}
\begin{tabular}{|p{3cm}|p{10cm}|} \hline
\centering {\bf PU-10} & El usuario busca una ley en lex.gal  \\ \hline\hline
\centering {\bf Versión} & 1.0 (08/06/2022) \\ \hline
\centering {\bf Dependencias} & FRQ-07. \\ \hline
\centering {\bf Descripción} &  El usuario busca cualquier ley en lex.gal, ya sea con texto de búsqueda o sin él. \\ \hline
\centering {\bf Criterio de aceptación} & Se muestran al usuario las leyes encontradas. \\ \hline
\centering {\bf Estado} & Superada \\ \hline
\end{tabular}
\caption{Prueba unitaria 10. El usuario busca una ley en lex.gal.}
\label{enlacePU10}
\end{center}
\end{table}

\begin{table}[H]
\begin{center}
\begin{tabular}{|p{3cm}|p{10cm}|} \hline
\centering {\bf PU-11} & Previsualizar una ley en lex.gal  \\ \hline\hline
\centering {\bf Versión} & 1.0 (08/06/2022) \\ \hline
\centering {\bf Dependencias} & FRQ-08. \\ \hline
\centering {\bf Descripción} &  El usuario intenta previsualizar una ley en lex.gal pinchando en el icono del ojo de las acciones de una ley de lex.gal. \\ \hline
\centering {\bf Criterio de aceptación} & Se muestra al usuario la ley correspondiente en una pestaña nueva. \\ \hline
\centering {\bf Estado} & Superada \\ \hline
\end{tabular}
\caption{Prueba unitaria 11. Previsualizar una ley en lex.gal.}
\label{enlacePU11}
\end{center}
\end{table}

\begin{table}[H]
\begin{center}
\begin{tabular}{|p{3cm}|p{10cm}|} \hline
\centering {\bf PU-12} & Cargar los datos de una ley de lex.gal  \\ \hline\hline
\centering {\bf Versión} & 1.0 (08/06/2022) \\ \hline
\centering {\bf Dependencias} & FRQ-09. \\ \hline
\centering {\bf Descripción} &  Al pinchar el botón de editar, se cargan los datos de la ley a editar en la página de edición. \\ \hline
\centering {\bf Criterio de aceptación} & Se muestra al usuario la ley correspondiente con sus datos en la página de edición. \\ \hline
\centering {\bf Estado} & Superada \\ \hline
\end{tabular}
\caption{Prueba unitaria 12. Cargar los datos de una ley de lex.gal.}
\label{enlacePU12}
\end{center}
\end{table}

\begin{table}[H]
\begin{center}
\begin{tabular}{|p{3cm}|p{10cm}|} \hline
\centering {\bf PU-13} & Proponer un cambio en la ley principal de lex.gal  \\ \hline\hline
\centering {\bf Versión} & 1.0 (08/06/2022) \\ \hline
\centering {\bf Dependencias} & FRQ-10. \\ \hline
\centering {\bf Descripción} &  El usuario propone un cambio sobre cualquier párrafo de la ley principal. \\ \hline
\centering {\bf Criterio de aceptación} & El cambio propuesto se añade a la lista de cambios, y se cambia el color de fondo del párrafo donde se introduce el cambio. \\ \hline
\centering {\bf Estado} & Superada \\ \hline
\end{tabular}
\caption{Prueba unitaria 13. Proponer un cambio en la ley principal de lex.gal.}
\label{enlacePU13}
\end{center}
\end{table}

\begin{table}[H]
\begin{center}
\begin{tabular}{|p{3cm}|p{10cm}|} \hline
\centering {\bf PU-14} & Descartar una selección de cambios de lex.gal  \\ \hline\hline
\centering {\bf Versión} & 1.0 (08/06/2022) \\ \hline
\centering {\bf Dependencias} & FRQ-10. \\ \hline
\centering {\bf Descripción} &  El usuario descarta una selección de cambios en la lista de cambios propuestos. \\ \hline
\centering {\bf Criterio de aceptación} & Se descarta la selección de cambios en la lista de cambios propuestos. \\ \hline
\centering {\bf Estado} & Superada \\ \hline
\end{tabular}
\caption{Prueba unitaria 14. Descartar una selección de cambios de lex.gal.}
\label{enlacePU14}
\end{center}
\end{table}

\begin{table}[H]
\begin{center}
\begin{tabular}{|p{3cm}|p{10cm}|} \hline
\centering {\bf PU-15} & Descartar todos los cambios propuestos a una ley de lex.gal  \\ \hline\hline
\centering {\bf Versión} & 1.0 (08/06/2022) \\ \hline
\centering {\bf Dependencias} & FRQ-10. \\ \hline
\centering {\bf Descripción} &  El usuario descarta todos los cambios en la lista de cambios propuestos. \\ \hline
\centering {\bf Criterio de aceptación} & Se descartan todos los cambios en la lista de cambios propuestos. \\ \hline
\centering {\bf Estado} & Superada \\ \hline
\end{tabular}
\caption{Prueba unitaria 15. Descartar todos los cambios propuestos a una ley de lex.gal.}
\label{enlacePU15}
\end{center}
\end{table}

\begin{table}[H]
\begin{center}
\begin{tabular}{|p{3cm}|p{10cm}|} \hline
\centering {\bf PU-16} & Añadir una anotación sobre un párrafo de la ley principal de lex.gal  \\ \hline\hline
\centering {\bf Versión} & 1.0 (08/06/2022) \\ \hline
\centering {\bf Dependencias} & FRQ-11. \\ \hline
\centering {\bf Descripción} &  El usuario añade una anotación sobre cualquier párrafo de la ley principal. \\ \hline
\centering {\bf Criterio de aceptación} & La anotación se añade a la lista de notas, y se cambia el color de fondo del párrafo donde se introduce el cambio. \\ \hline
\centering {\bf Estado} & Superada \\ \hline
\end{tabular}
\caption{Prueba unitaria 16. Añadir una anotación sobre un párrafo de la ley principal de lex.gal.}
\label{enlacePU16}
\end{center}
\end{table}

\begin{table}[H]
\begin{center}
\begin{tabular}{|p{3cm}|p{10cm}|} \hline
\centering {\bf PU-17} & Descartar una selección de anotaciones de una ley de lex.gal  \\ \hline\hline
\centering {\bf Versión} & 1.0 (08/06/2022) \\ \hline
\centering {\bf Dependencias} & FRQ-11. \\ \hline
\centering {\bf Descripción} &  El usuario descarta una selección de anotaciones en la lista de notas. \\ \hline
\centering {\bf Criterio de aceptación} & Se descarta la selección de anotaciones en la lista de notas. \\ \hline
\centering {\bf Estado} & Superada \\ \hline
\end{tabular}
\caption{Prueba unitaria 17. Descartar una selección de anotaciones de una ley de lex.gal.}
\label{enlacePU17}
\end{center}
\end{table}

\begin{table}[H]
\begin{center}
\begin{tabular}{|p{3cm}|p{10cm}|} \hline
\centering {\bf PU-18} & Descartar todas las anotaciones de una ley de lex.gal  \\ \hline\hline
\centering {\bf Versión} & 1.0 (08/06/2022) \\ \hline
\centering {\bf Dependencias} & FRQ-11. \\ \hline
\centering {\bf Descripción} &  El usuario descarta todas las anotaciones en la lista de notas. \\ \hline
\centering {\bf Criterio de aceptación} & Se descartan todas las anotaciones en la lista de notas. \\ \hline
\centering {\bf Estado} & Superada \\ \hline
\end{tabular}
\caption{Prueba unitaria 18. Descartar todas las anotaciones de una ley de lex.gal.}
\label{enlacePU18}
\end{center}
\end{table}

\begin{table}[H]
\begin{center}
\begin{tabular}{|p{3cm}|p{10cm}|} \hline
\centering {\bf PU-19} & Añadir un comentario sobre cualquier anotación  \\ \hline\hline
\centering {\bf Versión} & 1.0 (08/06/2022) \\ \hline
\centering {\bf Dependencias} & FRQ-11. \\ \hline
\centering {\bf Descripción} &  El usuario añade un comentario sobre una anotación. \\ \hline
\centering {\bf Criterio de aceptación} & El comentario es añadido a la anotación. \\ \hline
\centering {\bf Estado} & Superada \\ \hline
\end{tabular}
\caption{Prueba unitaria 19. Añadir un comentario sobre cualquier anotación.}
\label{enlacePU19}
\end{center}
\end{table}

\begin{table}[H]
\begin{center}
\begin{tabular}{|p{3cm}|p{10cm}|} \hline
\centering {\bf PU-20} & Cargar las leyes vinculadas a una ley de lex.gal  \\ \hline\hline
\centering {\bf Versión} & 1.0 (08/06/2022) \\ \hline
\centering {\bf Dependencias} & FRQ-12. \\ \hline
\centering {\bf Descripción} &  El usuario pincha en el apartado ``Leis vinc.'' en la pantalla de edición. \\ \hline
\centering {\bf Criterio de aceptación} & Se muestra la lista de leyes vinculadas a la ley. \\ \hline
\centering {\bf Estado} & Superada \\ \hline
\end{tabular}
\caption{Prueba unitaria 20. Cargar las leyes vinculadas a una ley de lex.gal.}
\label{enlacePU20}
\end{center}
\end{table}

\begin{table}[H]
\begin{center}
\begin{tabular}{|p{3cm}|p{10cm}|} \hline
\centering {\bf PU-21} & Añadir una ley vinculada a la lista de leyes vinculadas a una ley  \\ \hline\hline
\centering {\bf Versión} & 1.0 (08/06/2022) \\ \hline
\centering {\bf Dependencias} & FRQ-13. \\ \hline
\centering {\bf Descripción} &  El usuario añade una ley vinculada a la lista de leyes vinculadas de una ley principal. \\ \hline
\centering {\bf Criterio de aceptación} & Se añade la ley vinculada a la lista de leyes vinculadas. \\ \hline
\centering {\bf Estado} & Superada \\ \hline
\end{tabular}
\caption{Prueba unitaria 21. Añadir una ley vinculada a la lista de leyes vinculadas a una ley.}
\label{enlacePU21}
\end{center}
\end{table}

\begin{table}[H]
\begin{center}
\begin{tabular}{|p{3cm}|p{10cm}|} \hline
\centering {\bf PU-22} & Ir al párrafo donde se modifica una sección de la ley vinculada  \\ \hline\hline
\centering {\bf Versión} & 1.0 (08/06/2022) \\ \hline
\centering {\bf Dependencias} & FRQ-14. \\ \hline
\centering {\bf Descripción} &  El usuario pincha en la ley principal en un párrafo con fondo de color azul. \\ \hline
\centering {\bf Criterio de aceptación} & Se redirige al usuario a la parte de la ley vinculada donde se ha de realizar el cambio. \\ \hline
\centering {\bf Estado} & Superada \\ \hline
\end{tabular}
\caption{Prueba unitaria 22. Ir al párrafo donde se modifica una sección de la ley vinculada.}
\label{enlacePU22}
\end{center}
\end{table}

\begin{table}[H]
\begin{center}
\begin{tabular}{|p{3cm}|p{10cm}|} \hline
\centering {\bf PU-23} & Proponer un cambio sobre la ley vinculada  \\ \hline\hline
\centering {\bf Versión} & 1.0 (08/06/2022) \\ \hline
\centering {\bf Dependencias} & FRQ-14. \\ \hline
\centering {\bf Descripción} &  El usuario propone un cambio a una ley vinculada. \\ \hline
\centering {\bf Criterio de aceptación} & Se guarda el cambio en la ley vinculada y se muestra con un fondo de color verde. \\ \hline
\centering {\bf Estado} & Superada \\ \hline
\end{tabular}
\caption{Prueba unitaria 23. Proponer un cambio sobre la ley vinculada.}
\label{enlacePU23}
\end{center}
\end{table}

\begin{table}[H]
\begin{center}
\begin{tabular}{|p{3cm}|p{10cm}|} \hline
\centering {\bf PU-24} & Descartar un cambio sobre la ley vinculada  \\ \hline\hline
\centering {\bf Versión} & 1.0 (08/06/2022) \\ \hline
\centering {\bf Dependencias} & FRQ-14. \\ \hline
\centering {\bf Descripción} &  El usuario descarta un cambio sobre una ley vinculada. \\ \hline
\centering {\bf Criterio de aceptación} & Se descarta el cambio en la ley vinculada y se quita el color de fondo. \\ \hline
\centering {\bf Estado} & Superada \\ \hline
\end{tabular}
\caption{Prueba unitaria 24. Descartar un cambio sobre la ley vinculada.}
\label{enlacePU24}
\end{center}
\end{table}

\begin{table}[H]
\begin{center}
\begin{tabular}{|p{3cm}|p{10cm}|} \hline
\centering {\bf PU-25} & Eliminar una ley de lex.gal  \\ \hline\hline
\centering {\bf Versión} & 1.0 (08/06/2022) \\ \hline
\centering {\bf Dependencias} & FRQ-15. \\ \hline
\centering {\bf Descripción} &  El usuario pincha en ``Eliminar de lex.gal''. \\ \hline
\centering {\bf Criterio de aceptación} & Se elimina la ley de lex.gal. \\ \hline
\centering {\bf Estado} & Superada \\ \hline
\end{tabular}
\caption{Prueba unitaria 25. Eliminar una ley de lex.gal.}
\label{enlacePU25}
\end{center}
\end{table}

\begin{table}[H]
\begin{center}
\begin{tabular}{|p{3cm}|p{10cm}|} \hline
\centering {\bf PU-26} & Validar y publicar una ley de lex.gal  \\ \hline\hline
\centering {\bf Versión} & 1.0 (08/06/2022) \\ \hline
\centering {\bf Dependencias} & FRQ-10, FRQ-11, FRQ-12, FRQ-13, FRQ-14. \\ \hline
\centering {\bf Descripción} &  El usuario pincha en ``Validar e publicar''. \\ \hline
\centering {\bf Criterio de aceptación} & Se almacena la ley con todos los cambios, anotaciones, cambios sobre leyes vinculadas, leyes vinculadas a la ley y datos de cabecera editados en lex.gal. El estado de la ley, y el estado de sus leyes vinculadas, pasa a ser de Validada. \\ \hline
\centering {\bf Estado} & Superada \\ \hline
\end{tabular}
\caption{Prueba unitaria 26. Validar y publicar una ley de lex.gal.}
\label{enlacePU26}
\end{center}
\end{table}

\begin{table}[H]
\begin{center}
\begin{tabular}{|p{3cm}|p{10cm}|} \hline
\centering {\bf PU-27} & Guardar como borrador una ley de lex.gal  \\ \hline\hline
\centering {\bf Versión} & 1.0 (08/06/2022) \\ \hline
\centering {\bf Dependencias} & FRQ-10, FRQ-11, FRQ-12, FRQ-13, FRQ-14. \\ \hline
\centering {\bf Descripción} &  El usuario pincha en ``Gardar como borrador''. \\ \hline
\centering {\bf Criterio de aceptación} & Se almacena la ley con todos los cambios, anotaciones, cambios sobre leyes vinculadas, leyes vinculadas a la ley y datos de cabecera editados en lex.gal. El estado de la ley, y el estado de sus leyes vinculadas, pasa a ser de Borrador. \\ \hline
\centering {\bf Estado} & Superada \\ \hline
\end{tabular}
\caption{Prueba unitaria 27. Guardar como borrador una ley de lex.gal.}
\label{enlacePU27}
\end{center}
\end{table}