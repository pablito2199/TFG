\section{Conclusiones}

La redacción de textos jurídicos requiere un trabajo muy laborioso, donde en todo momento se ha de ser muy preciso con las palabras empleadas. Cualquier afirmación hecha en una ley no puede dejar lugar a dudas, pues podrían quedar ambigüedades de las que cualquier persona podría aprovecharse de la redacción de una ley para determinadas acciones.
\\

Gracias a este TFG se han adquirido una gran cantidad de conocimientos acerca de la redacción de textos jurídicos, además de como deben ser redactados, modificados, y como han de ser filtrados en las búsquedas.
\\

En cuanto a las tecnologías empleadas, se ha adquirido un alto grado de conocimiento en tecnologías relacionadas con el desarrollo web. Si bien se habían empleado anteriormente algunos lenguajes como Java, HTML o JavaScript, en ningún momento se había procedido a la realización de una aplicación de forma completa, tanto la parte del servidor, como el cliente, y la propia base de datos.
\\

Se han adquirido conocimientos en React.js, uno de los frameworks de JavaScript más utilizados hoy en día por su gran rendimiento y calidad a la hora de realizar {\it frontend}. En cuanto al estilizado de las páginas, se ha aprendido a utilizar Tailwind, un framework de CSS muy popular en los últimos años.
\\

En cuanto a la parte del servidor y la base de datos, se han ampliado múltiples conocimientos en Spring Boot y MongoDB. También se ha comprendido como documentar correctamente la API de esta aplicación mediante el uso de Swagger.
\\

Para finalizar, cabe destacar que si bien la aplicación está diseñada para utilizar por parte de los usuarios de lex.gal, realizando una serie de pequeñas modificaciones podría llegarse a destinar el proyecto a cualquier aplicación donde fuese necesaria la edición de documentos de forma colaborativa entre usuarios.