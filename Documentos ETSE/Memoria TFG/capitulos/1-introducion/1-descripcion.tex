\section{Descripción del problema}
La creación de contenido constituye uno de los problemas más importantes a los que se enfrentan muchos dominios de aplicación, entre ellos, la educación/formación, generación de noticias, producción científica o la legislación. En estos dominios, el objetivo en la creación de los contenidos es doble: por una parte, creación de contenido de calidad, entendida como la fiabilidad, de modo que se minimicen los errores y que, si existen, sean sencillos de detectar; y, por otra parte, adaptación del contenido a los usuarios que los van a consumir, proporcionando herramientas que faciliten la comprensión de dicho contenido por parte de los usuarios.
\\

Este Trabajo Fin de Grado (TFG) está centrado en el ámbito de la legislación y, más específicamente, en la creación y el consumo de textos legislativos (leyes o decretos) por parte de especialistas en el ámbito del dominio. Por lo tanto, la principal restricción es la generación de contenido fiable, ya que los usuarios a los que van destinados los documentos son del mismo ámbito y nivel de especialización y/o formación que los usuarios que los crean. Esta es una característica clave que se tendrá muy en cuenta a la hora de diseñar el sistema.
\\

Por otra parte, desde hace varias décadas se han propuesto una gran cantidad de estrategias o métodos para garantizar la fiabilidad de los textos, pero quizás la edición colaborativa de documentos ha sido la que ha tenido un mayor éxito y aceptación por parte de los usuarios \cite{collaborative}. Un ejemplo claro de ello es la Wikipedia \cite{wikipedia}, en la que cualquier usuario registrado puede proponer y generar cambios que, no obstante, deberán ser aceptados por los administradores a partir de los comentarios que se introducen en un panel de discusión al que podrán acceder todos los usuarios que hayan participado en la edición del contenido en cuestión \cite{knowledge}. Este tipo de edición ha proporcionado a la Wikipedia una gran calidad y fiabilidad como modo de transmisión de conocimiento \cite{trustworthiness}.
\\

En este TFG se desarrollará una herramienta colaborativa para la edición de documentos en el ámbito de la legislación, {\bf lex.gal}. En este contexto, la colaboración se entiende desde dos puntos de vista: por una parte, la edición por parte de varios usuarios, de modo que las modificaciones de unos se comuniquen a otros con el objeto de que puedan confirmarlas y/o discutir si esa modificación es pertinente o no; y, por otra parte, la edición simultánea de los documentos, de forma que los usuarios puedan ir viendo en una misma sesión las modificaciones y/o correcciones que realizan otros usuarios. La implementación de este tipo de edición colaborativa tendrá en cuenta las características de los documentos legislativos, que tienen una estructura muy definitiva, a modo de artículos y apartados, en los que, además, se cruza mucha información de otros textos legislativos. Además, también se permitirá realizar una búsqueda de documentos en el DOG para poder importar la ley/decreto deseado, y así poder proceder a la edición del propio documento.