\section{Descripción del problema}

Los portales de consulta de legislación son extremadamente importantes en cualquier ámbito, ya que permiten consultar cualquier resolución o ley publicada por parte de todos los públicos. Actualmente existen portales donde poder consultar las leyes publicadas por parte de las instituciones públicas, como puede ser el DOG, a nivel gallego, o el BOE, a nivel nacional. En todos los casos, debe ser necesario poder acceder a la ley consolidada, es decir, la última versión de la ley donde poder ver si se han realizado cambios, correcciones, añadido anotaciones, etc. Por ejemplo, en el BOE existe un portal \cite{boe} donde poder consultar dichas leyes. Es por ello que a nivel autonómico es necesaria la existencia de un sistema donde publicar las leyes consolidadas.
\\

En cuanto a la creación de contenido, esta constituye uno de los problemas más importantes a los que se enfrentan muchos dominios de aplicación, entre ellos, la educación, la formación, generación de noticias, producción científica o la legislación. En estos dominios, el objetivo en la creación de los contenidos es doble: por una parte, creación de contenido de calidad, entendida como la fiabilidad, de modo que se minimicen los errores y que, si existen, sean sencillos de detectar; y, por, otra parte, la adaptación del contenido a los usuarios que los van a consumir, proporcionando herramientas que faciliten la comprensión de dicho contenido por parte de los usuarios de la aplicación.
\\

Este Trabajo Fin de Grado (TFG) está centrado en el ámbito de la legislación y, más específicamente, en la creación y el consumo de textos legislativos (leyes o decretos) por parte de especialistas en el ámbito del dominio. Por lo tanto, la principal restricción es la generación de contenido fiable, ya que los usuarios a los que van destinados los documentos son del mismo ámbito y nivel de especialización y/o formación que los usuarios que los crean. Esta es una característica clave que se tendrá muy en cuenta a la hora de diseñar el sistema.
\\

Por otra parte, desde hace varias décadas se han propuesto una gran cantidad de estrategias o métodos para garantizar la fiabilidad de los textos, pero quizás la edición colaborativa de documentos ha sido la que ha tenido un mayor éxito y aceptación por parte de los usuarios \cite{collaborative}. Un ejemplo claro de ello es la Wikipedia \cite{wikipedia}, en la que cualquier usuario registrado puede proponer y generar cambios que, no obstante, deberán ser aceptados por los administradores a partir de los comentarios que se introducen en un panel de discusión al que podrán acceder todos los usuarios que hayan participado en la edición del contenido en cuestión \cite{knowledge}. Este tipo de edición ha proporcionado a la Wikipedia una gran calidad y fiabilidad como modo de transmisión de conocimiento \cite{trustworthiness}.
\\

El contexto en el que surge este TFG es el desarrollo de una herramienta para el portal de publicación de leyes consolidadas a nivel gallego, lex.gal \cite{lexgal}. La solución que se utiliza actualmente en este portal resulta muy tediosa para los usuarios. Algunos problemas que se presentan son la escasa usabilidad, el trabajo que deben realizar los usuarios manualmente es excesivo, siendo este susceptible a errores; necesidad de publicar la ley para poder consultar que los cambios se aplican correctamente (siendo necesario borrar la ley junto con sus cambios del portal en caso de haber cometido un fallo), etc.
\\

Para solucionar todos los problemas posibles que se presentan en la aplicación que se utiliza hoy en día, en este TFG se desarrollará una herramienta colaborativa que permite la edición de documentos en el ámbito de la legislación. En este contexto, la colaboración se entiende como la edición por parte de varios usuarios, de modo que las modificaciones de unos se comuniquen a otros con el objeto de que puedan confirmarlas y/o discutir si esa modificación es pertinente o no. La implementación de este tipo de edición colaborativa tendrá en cuenta las características de los documentos legislativos, que tienen una estructura muy definitiva, a modo de artículos y apartados, en los que, además, se cruza mucha información de otros textos legislativos. En todo caso, la edición por parte de varios usuarios no será simultánea. Además, también se permitirá realizar una búsqueda de documentos en el DOG para poder importar las leyes deseadas, y así poder proceder a la edición del propio documento.