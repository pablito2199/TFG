\section{Estructura del documento}
La presente memoria recoge toda la documentación del proyecto realizado en este TFG. Esta se encuentra dividida en una serie de capítulos, los cuales pueden ser resumidos de la siguiente forma:
\begin{itemize}
\item {\bf Capítulo 1: Introducción}. En este capítulo se aborda la descripción del problema y motivación del proyecto, los objetivos generales del mismo y la estructura de la memoria.
\item {\bf Capítulo 2: Especificación de Requisitos}. Esta sección del documento define los requisitos existentes que garantizarán la aceptación del proyecto por parte de los usuarios especialistas en textos jurídicos.
\item {\bf Capítulo 3: Diseño}. El diseño incluye el propio diseño de la aplicación y la comunicación entre sus diferentes componentes, la arquitectura del sistema, diagramas UML y el diseño de la interfaz gráfica.
\item {\bf Capítulo 4: Pruebas}. Este capítulo contiene un informe acerca de las pruebas unitarias, de integración y validación realizadas en el sistema.
\item {\bf Capítulo 5: Conclusiones y posibles ampliaciones}. En este último capítulo se extraen diferentes conclusiones obtenidas tras la realización del TFG, así como posibles ampliaciones que se podrán realizar en un futuro.
\item {\bf Apéndices:}
\begin{itemize}
\item \underline{Apéndice A: Manuales técnicos}. Incluye un manual que contiene una explicación acerca de cómo instalar las diferentes dependencias de la aplicación, así como una guía para poder desplegarla correctamente en cualquier máquina.
\item \underline{Apéndice B: Manuales de usuario}. En este apéndice se describe cómo un usuario podrá utilizar la aplicación web que ha sido desarrollada.
\item \underline{Apéndice C: Licencia}. Se trata de la licencia de uso del software y documentación del propio TFG.
\end{itemize}
\end{itemize}