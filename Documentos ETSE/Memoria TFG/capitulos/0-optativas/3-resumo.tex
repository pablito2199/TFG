\pagestyle{plain}
\chapter*{Resumen}
Este trabajo de fin de grado aborda el desarrollo de una herramienta colaborativa que permitirá la creación y edición de documentos por parte de varios usuarios. Esta herramienta será integrada en una plataforma de consulta de legislación a nivel gallego, que facilitará la importación de legislación desde el Diario Oficial de Galicia (DOG) \cite{dog} y su mantenimiento con la aplicación en las sucesivas modificaciones que se vayan realizando.
\\

De esta forma, los usuarios podrán importar documentos a la herramienta colaborativa a partir de leyes importadas del DOG. Por una parte, se permite la búsqueda de leyes y normas encontradas en el DOG filtrando según los parámetros deseados. Adicionalmente, podrá verse una previsualización en la página oficial del DOG. Dichos documentos encontrados en el DOG podrán ser importados a lex.gal, donde se podrá proceder a su edición. En cuanto a la edición, se ofrece la posibilidad de proponer cambios a la norma que se va a editar, manipular las leyes vinculadas a esta, así como escribir posibles notas aclaratorias. Todo ello será gestionado por usuarios previamente registrados.