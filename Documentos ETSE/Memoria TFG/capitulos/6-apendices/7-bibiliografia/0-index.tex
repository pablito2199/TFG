\markboth{BIBLIOGRAFÍA}{BIBLIOGRAFÍA}
\addcontentsline{toc}{chapter}{Bibliografía}


\begin{thebibliography}{99}

\bibitem{dog}{\it Inicio - Diario Oficial de Galicia}. Disponible en {\it \url{https://www.xunta.gal/diario-oficial-galicia/portalPublicoHome.do?lang=gl}}. Consultado el 29 de abril de 2022.

\bibitem{boe}{\it Legislación consolidada: Información y ayuda}. BOE. Disponible en {\it \url{https://www.boe.es/buscar/ayudas/legislacion_actualizada.php#faq001i}}. Consultado el 15 de junio de 2022.

\bibitem{lexgal}{\it Inicio - Dereito consolidado do galego - lex.gal}. Disponible en {\it \url{https://www.lex.gal/portada}}. Consultado el 29 de abril de 2022.

\bibitem{collaborative} Bani-Salameh, H., \& Jeffery, C. (2014). Collaborative and social development environments: a literature review. International Journal of Computer Applications in Technology, 49(2), 89-103.

\bibitem{wikipedia}{\it Wikipedia, la enciclopedia libre}. Wikipedia. Disponible en {\it \url{http://es.wikipedia.es}}. Consultado el 2 de mayo de 2022.

\bibitem{knowledge} Oeberst, A., Halatchliyski, I., Kimmerle, J., \& Cress, U. (2014). Knowledge construction in Wikipedia: A systemic-constructivist analysis. Journal of the Learning Sciences, 23(2), 149-176.

\bibitem{trustworthiness} Amina, W., \& Warraich, N. F. (2022). Use and trustworthiness of Wikipedia information: students’ perceptions and reflections. Digital Library Perspectives, 38(1), 16-32..

\bibitem{rem}{\it Herramienta REM}. Departamento de lenguajes y sistemas informáticos. Universidad de Sevilla. . Disponible en {\it \url{http://www.lsi.us.es/descargas/descarga_programas.php?id=3}}. Consultado el 6 de junio de 2022.

\bibitem{requisitos}{\it Requisitos (sistemas)}. Artículo de la Wikipedia ({\it \url{https://es.wikipedia.org/wiki/Requisito_(sistemas)}}). Consultado el 6 de junio de 2022.

\bibitem{casodeuso}{\it Caso de uso}. Artículo de la Wikipedia ({\it \url{https://es.wikipedia.org/wiki/Caso_de_uso}}). Consultado el 6 de junio de 2022.

\bibitem{frontend}{\it Qué es Frontend y Backend: diferencias y características}. Maldeadora. Platzi. Disponible en {\it \url{https://platzi.com/blog/que-es-frontend-y-backend/?utm_source=google&utm_medium=paid&utm_campaign=17446514363&utm_adgroup=&utm_content=&gclid=CjwKCAjwnZaVBhA6EiwAVVyv9AqeKxLkNDaygNbujqsVWey4a4qDcDWaU21chBlbgBHEIvbpIHbnyhoC0dAQAvD_BwE&gclsrc=aw.ds}}. Consultado el 12 de junio de 2022.

\bibitem{react}{\it React - Una biblioteca de JavaScript para construir interfaces de usuario}. Disponible en {\it \url{https://es.reactjs.org/}}. Consultado el 29 de abril de 2022.

\bibitem{tailwind}{\it Tailwind CSS - Rapidly build modern websites without ever leaving your HTML}. Disponible en {\it \url{https://tailwindcss.com/}}. Consultado el 29 de abril de 2022.

\bibitem{css} {\it ¿Qué es CSS? - CSS en español}. Lenguaje CSS ({\it \url{https://lenguajecss.com/css/introduccion/que-es-css/}}). Consultado el 12 de junio de 2022.

\bibitem{nodejs}{\it Node.js}. Disponible en {\it \url{https://nodejs.org/es/}}. Consultado el 3 de junio de 2022.

\bibitem{npm}{\it NPM, Node package manager}. Disponible en {\it \url{https://www.npmjs.com/}}. Consultado el 2 de mayo de 2022.

\bibitem{reactdiffviewer}{\it React Diff Viewer}. Disponible en {\it \url{https://www.npmjs.com/package/react-diff-viewer}}. Consultado el 9 de junio de 2022.

\bibitem{heroicons}{\it Heroicons}. Disponible en {\it \url{https://heroicons.com/}}. Consultado el 9 de junio de 2022.

\bibitem{materialui}{\it MUI:The React component library you always wanted}. Disponible en {\it \url{https://mui.com/}}. Consultado el 9 de junio de 2022.

\bibitem{spring}{\it Spring Boot}. Disponible en {\it \url{https://spring.io/projects/spring-boot}}. Consultado el 29 de abril del 2022.

\bibitem{gradle}{\it Gradle Build Tool}. Disponible en {\it \url{https://gradle.org/}}. Consultado el 9 de junio del 2022.

\bibitem{mongodb}{\it MongoDB: La Plataforma De Datos Para Aplicaciones - MongoDB}. Disponible en {\it \url{https://www.mongodb.com/es}}. Consultado el 2 de mayo de 2022.

\bibitem{nosql}{\it Qué es una base de datos NoSQL y por qué su empresa debería usarla para mejorar el rendimiento y los datos, reducir los costos y aumentar la velocidad de llegada al mercado}. Disponible en {\it \url{https://www.rackspace.com/es/library/what-is-a-nosql-database#:~:text=NoSQL%20se%20refiere%20a%20una,formatos%20que%20no%20sean%20tablas.}}. Consultado el 12 de junio de 2022.

\bibitem{swagger}{\it API documentation \& Design Tools for Teams - Swagger}. Disponible en {\it \url{https://swagger.io/}}. Consultado el 2 de mayo de 2022.

\bibitem{rest}{\it Arquitectura REST: Concepto y fundamentos}. GaussWebApp. Disponible en {\it \url{https://gausswebapp.com/arquitectura-rest.html}}. Consultado el 16 de junio de 2022.

\bibitem{evergreen}{\it ¿Qué es el navegador de hoja perenne?}. Techopedia. Disponible en {\it \url{https://es.theastrologypage.com/evergreen-browser}}. Consultado el 16 de junio de 2022.

\bibitem{logicanegocio}{\it Lógica de negocio}. Conecta Software. Disponible en {\it \url{https://conectasoftware.com/glosario/logica-de-negocio/}}. Consultado el 16 de junio de 2022.

\bibitem{jwt}{\it JSON Web Tokens - jwt.io}. Disponible en {\it \url{https://jwt.io/}}. Consultado el 6 de junio de 2022.

\bibitem{soa}{\it Arquitectura orientada a servicios}. Artículo de la Wikipedia ({\it \url{https://es.wikipedia.org/wiki/Arquitectura_orientada_a_servicios}}). Consultado el 9 de junio de 2022.

\bibitem{servlet}{\it Introducción a los servlets}. Manual Web. ({\it \url{https://www.manualweb.net/javaee/introduccion-servlets/}}). Consultado el 20 de junio de 2022.

\bibitem{django}{\it Django}. The web framework for perfectionists with deadlines. ({\it \url{https://www.djangoproject.com/}}). Consultado el 20 de junio de 2022.

\bibitem{spa}{\it Arquitectura de un SPA}. Desarrollo de aplicaciones web. ({\it \url{https://juanda.gitbooks.io/webapps/content/spa/arquitectura_de_un_spa.html}}). Consultado el 20 de junio de 2022.

\bibitem{composite}{\it Composite}. Refactoring Guru. ({\it \url{https://refactoring.guru/es/design-patterns/composite}}). Consultado el 12 de junio de 2022.

\bibitem{frontcontroller}{\it Front Controller}. Tutorials Point. ({\it \url{https://www.tutorialspoint.com/design_pattern/front_controller_pattern.htm}}). Consultado el 20 de junio de 2022.

\bibitem{state}{\it State}. Refactoring Guru. ({\it \url{https://refactoring.guru/es/design-patterns/state}}). Consultado el 20 de junio de 2022.

\bibitem{hooks}{\it Hooks Pattern}. Patterns. ({\it \url{https://www.patterns.dev/posts/hooks-pattern/}}). Consultado el 20 de junio de 2022.

\bibitem{mockup}{\it ¿Qué es un Mock Up?}. Claudia Bravo. Estudioka. ({\it \url{https://estudioka.es/que-es-un-mock-up/}}). Consultado el 19 de junio de 2022.

\bibitem{pruebaunitaria}{\it Prueba unitaria}. Artículo de la Wikipedia ({\it \url{https://es.wikipedia.org/wiki/Prueba_unitaria}}). Consultado el 7 de junio de 2022.

\bibitem{pruebaintegracion}{\it Prueba de integración}. Artículo de la Wikipedia ({\it \url{https://es.wikipedia.org/wiki/Prueba_de_integraci%C3%B3n}}). Consultado el 7 de junio de 2022.

\bibitem{sus}{\it Cómo medir la usabilidad con un SUS}. Cris Busquets. Disponible en {\it \url{https://www.uifrommars.com/como-medir-usabilidad-que-es-sus/}}. Consultado el 7 de junio de 2022.

\bibitem{nielsen}{\it Los 10 principios de usabilidad de Jakob Nielsen}. Beatriz Allas Miguelsanz. Disponible en {\it \url{https://profile.es/blog/los-10-principios-de-usabilidad-web-de-jakob-nielsen/}}. Consultado el 7 de junio de 2022.

\bibitem{infoherramientas}{\it Información sobre herramientas}. Artículo de la Wikipedia ({\it \url{https://es.wikipedia.org/wiki/Informaci%C3%B3n_sobre_herramientas}}). Consultado el 7 de junio de 2022.

\bibitem{websocket}{\it WebSocket}. Artículo de la Wikipedia ({\it \url{https://es.wikipedia.org/wiki/WebSocket}}). Consultado el 8 de junio de 2022.

\bibitem{intellij}{\it IntelliJ IDEA: el IDE de Java eficaz y ergonómico de JetBrains}. Disponible en {\it \url{https://www.jetbrains.com/es-es/idea/}}. Consultado el 3 de junio de 2022.

\bibitem{jdk}{\it Java Downloads | Oracle}. Disponible en {\it \url{https://www.oracle.com/java/technologies/downloads/#jdk18-windows}}. Consultado el 5 de junio de 2022.

\bibitem{cmd}{\it Símbolo del sistema de Windows}. Artículo de la Wikipedia ({\it \url{https://es.wikipedia.org/wiki/S%C3%ADmbolo_del_sistema_de_Windows}}). Consultado el 24 de mayo de 2022.

\bibitem{mit}{\it The MIT License - Open Source Initiative}. Disponible en {\it \url{https://opensource.org/licenses/MIT}}. Consultado el 29 de abril de 2022.
\end{thebibliography}