\chapter{Manuales técnicos}

En este apéndice se puede consultar como desplegar la aplicación, tanto el cliente, como el servidor, y la propia base de datos. Cabe recordar que se debe tener acceso a Internet en todo momento para un correcto funcionamiento de la herramienta, puesto que la página de búsquedas localizará los documentos contra la API del DOG \cite{dog}. Otro detalle a tener en cuenta es que la aplicación se ha probado mayoritariamente en un entorno con un sistema Windows 10.

\section{Manual de despliegue}

Para poder ejecutar la aplicación en cualquier ordenador, se ha de contar con las siguientes herramientas instaladas:
\begin{itemize}
\item {\bf Node.js \cite{nodejs}}: entorno donde se desplegará el cliente. Permite la instalación de {\bf npm \cite{npm}}, que se encarga de gestionar los paquetes de Node.js
\item {\bf IntelliJ Idea \cite{intellij}} o cualquier {\bf Java SE Development Kit (JDK) \cite{jdk} versión 15.0 o superior}: necesario para poder ejecutar los archivos que contienen el servidor de la aplicación.
\item {\bf MongoDB \cite{mongodb}}: sistema de base de datos NoSQL necesario para poder emplear la base de datos de la aplicación.
\end{itemize}

Una vez hemos instalado estos tres programas, procedemos al despliegue. Para poder proceder al despligue de la {\bf capa de servicios} (elegir una de las dos opciones):
\\

- En caso de contar con un JDK instalado en nuestro ordenador:
\begin{enumerate}
\item Acceder al directorio {\it Servidor/build/libs} en la carpeta del proyecto.
\item Ejecutar el archivo {\it TFG-Servidor.jar}.
\end{enumerate}

- En caso de contar con Intellij Idea en nuestro ordenador:
\begin{enumerate}
\item Ejecutar la aplicación de IntelliJ Idea en nuestra máquina.
\item Pinchar en {\it Open Project} y seleccionamos la carpeta {\it Servidor} que se encuentra dentro del directorio proyecto.
\item Ejecutar el archivo {\it Application.java}.
\item Una vez se ha ejecutado, esperamos unos segundos para que termine de compilar. El servicio será desplegado en el puerto 8000 (acceso mediante {\it \url{http://localhost:8000/}}).
\end{enumerate}

En cuanto al despliegue de la parte del {\bf cliente}:
\begin{enumerate}
\item Abrir cualquier terminal disponible en el ordenador (por ejemplo, {\it Símbolos del Sistema de Windows (CMD)} \cite{cmd}).
\item Acceder a la carpeta {\it Cliente} dentro del directorio del proyecto.
\item Ejecutar el comando {\it npm install}.
\item Ejecutar el comando {\it npm start}.
\item Se abrirá una pestaña en su navegador predeterminado. Este proceso puede demorarse unos minutos. En caso de no abrirse automáticamente, puede acceder manualmente accediendo al puerto 3000, {\it \url{http://localhost:3000/}}.
\end{enumerate}

\section{Versiones de las herramientas recomendadas}
Para un correcto funcionamiento de la aplicación, se recomienda utilizar las siguientes herramientas:
\begin{itemize}
\item {\bf Node.js} (16.15.0 (64 bits)).
\item {\bf IntelliJ} Idea (2022.1.1 (64 bits)).
\item {\bf Java SE Development Kit (JDK)} (18.0.1.1 (64 bits)).
\item {\bf MongoDB Community} (5.0.8 (64 bits)).
\item {\bf Navegador Google Chrome} (Versión 101.0.4951.67 (Build oficial) (64 bits)).
\item {\bf Sistema Operativo Windows 10} (Versión 21H2 (64 bits)).
\item {\bf Resolución de pantalla} de 15.6" (1920x1080).
\end{itemize}