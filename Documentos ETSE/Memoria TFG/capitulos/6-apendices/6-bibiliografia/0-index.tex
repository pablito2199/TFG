\markboth{BIBLIOGRAFÍA}{BIBLIOGRAFÍA}
\addcontentsline{toc}{chapter}{Bibliografía}


\begin{thebibliography}{99}

\bibitem{dog}{\it Inicio - Diario Oficial de Galicia}. Disponible en {\it \url{https://www.xunta.gal/diario-oficial-galicia/portalPublicoHome.do?lang=gl}}. Consultado el 29 de abril de 2022.

\bibitem{lexgal}{\it Inicio - Dereito consolidado do galego - lex.gal}. Disponible en {\it \url{https://www.lex.gal/portada}}. Consultado el 29 de abril de 2022.

\bibitem{collaborative} Bani-Salameh, H., \& Jeffery, C. (2014). Collaborative and social development environments: a literature review. {\it International journal of computer applications in technology, 49(2)}, 89-103.

\bibitem{wikipedia} Wikipedia, la enciclopedia libre. Disponible en {\it \url{http://es.wikipedia.es}}. Consultado el 2 de mayo de 2022.

\bibitem{knowledge} Oeberst, A., Halatchliyski, I., Kimmerle, J., \& Cress, U. (2014). Knowledge construction in Wikipedia: A systemic-constructivist analysis. Journal of the Learning Sciences, 23(2), 149-176.

\bibitem{trustworthiness} Amina, W., \& Warraich, N. F. (2021). Use and trustworthiness of Wikipedia information: students’ perceptions and reflections. Digital Library Perspectives.

\bibitem{rem}{\it Departamento de lenguajes y sistemas informáticos. Universidad de Sevilla. Herramienta REM.}. Disponible en {\it \url{http://www.lsi.us.es/descargas/descarga_programas.php?id=3}}. Consultado el 6 de junio de 2022.

\bibitem{requisitos} Requisitos (sistemas). Artículo de la Wikipedia ({\it \url{https://es.wikipedia.org/wiki/Requisito_(sistemas)}}). Consultado el 6 de junio de 2022.

\bibitem{casodeuso} Caso de uso. Artículo de la Wikipedia ({\it \url{https://es.wikipedia.org/wiki/Caso_de_uso}}). Consultado el 6 de junio de 2022.

\bibitem{nodejs}{\it Node.js}. Disponible en {\it \url{https://nodejs.org/es/}}. Consultado el 3 de junio de 2022.

\bibitem{mongodb}{\it MongoDB: La Plataforma De Datos Para Aplicaciones - MongoDB}. Disponible en {\it \url{https://www.mongodb.com/es}}. Consultado el 2 de mayo de 2022.

\bibitem{spring}{\it Spring Boot}. Disponible en {\it \url{https://spring.io/projects/spring-boot}}. Consultado el 29 de abril del 2022.

\bibitem{gradle}{\it Gradle Build Tool}. Disponible en {\it \url{https://gradle.org/}}. Consultado el 9 de junio del 2022.

\bibitem{swagger}{\it API documentation \& Design Tools for Teams - Swagger}. Disponible en {\it \url{https://swagger.io/}}. Consultado el 2 de mayo de 2022.

\bibitem{jwt}{\it JSON Web Tokens - jwt.io}. Disponible en {\it \url{https://jwt.io/}}. Consultado el 6 de junio de 2022.

\bibitem{react}{\it React - Una biblioteca de JavaScript para construir interfaces de usuario}. Disponible en {\it \url{https://es.reactjs.org/}}. Consultado el 29 de abril de 2022.

\bibitem{tailwind}{\it Tailwind CSS - Rapidly build modern websites without ever leaving your HTML}. Disponible en {\it \url{https://tailwindcss.com/}}. Consultado el 29 de abril de 2022.

\bibitem{npm}{\it NPM, Node package manager}. Disponible en {\it \url{https://www.npmjs.com/}}. Consultado el 2 de mayo de 2022.

\bibitem{reactdiffviewer}{\it React Diff Viewer}. Disponible en {\it \url{https://www.npmjs.com/package/react-diff-viewer}}. Consultado el 9 de junio de 2022.

\bibitem{heroicons}{\it Heroicons}. Disponible en {\it \url{https://heroicons.com/}}. Consultado el 9 de junio de 2022.

\bibitem{materialui}{\it MUI:The React component library you always wanted}. Disponible en {\it \url{https://mui.com/}}. Consultado el 9 de junio de 2022.

\bibitem{soa} Arquitectura orientada a servicios. Artículo de la Wikipedia ({\it \url{https://es.wikipedia.org/wiki/Arquitectura_orientada_a_servicios}}). Consultado el 9 de junio de 2022.

\bibitem{flux}{\it ¿Qué es Flux? Entendiendo su arquitectura}. Carlos Azaustre. Disponible en {\it \url{https://carlosazaustre.es/como-funciona-flux}}. Consultado el 9 de junio de 2022.

\bibitem{builder} Builder (patrón de diseño). Artículo de la Wikipedia ({\it \url{https://es.wikipedia.org/wiki/Builder_(patr%C3%B3n_de_dise%C3%B1o)}}). Consultado el 9 de junio de 2022.

\bibitem{composite} Composite (patrón de diseño). Artículo de la Wikipedia ({\it \url{https://es.wikipedia.org/wiki/Composite_(patr%C3%B3n_de_dise%C3%B1o)}}). Consultado el 9 de junio de 2022.

\bibitem{pruebaunitaria} Prueba unitaria. Artículo de la Wikipedia ({\it \url{https://es.wikipedia.org/wiki/Prueba_unitaria}}). Consultado el 7 de junio de 2022.

\bibitem{pruebaintegracion} Prueba de integración. Artículo de la Wikipedia ({\it \url{https://es.wikipedia.org/wiki/Prueba_de_integraci%C3%B3n}}). Consultado el 7 de junio de 2022.

\bibitem{sus}{\it Cómo medir la usabilidad con un SUS}. Cris Busquets. Disponible en {\it \url{https://www.uifrommars.com/como-medir-usabilidad-que-es-sus/}}. Consultado el 7 de junio de 2022.

\bibitem{nielsen}{\it Los 10 principios de usabilidad de Jakob Nielsen}. Beatriz Allas Miguelsanz. Disponible en {\it \url{https://profile.es/blog/los-10-principios-de-usabilidad-web-de-jakob-nielsen/}}. Consultado el 7 de junio de 2022.

\bibitem{infoherramientas} Información sobre herramientas. Artículo de la Wikipedia ({\it \url{https://es.wikipedia.org/wiki/Informaci%C3%B3n_sobre_herramientas}}). Consultado el 7 de junio de 2022.

\bibitem{websocket} WebSocket. Artículo de la Wikipedia ({\it \url{https://es.wikipedia.org/wiki/WebSocket}}). Consultado el 8 de junio de 2022.

\bibitem{intellij}{\it IntelliJ IDEA: el IDE de Java eficaz y ergonómico de JetBrains}. Disponible en {\it \url{https://www.jetbrains.com/es-es/idea/}}. Consultado el 3 de junio de 2022.

\bibitem{jdk}{\it Java Downloads | Oracle}. Disponible en {\it \url{https://www.oracle.com/java/technologies/downloads/#jdk18-windows}}. Consultado el 5 de junio de 2022.

\bibitem{cmd} Símbolo del sistema de Windows. Artículo de la Wikipedia ({\it \url{https://es.wikipedia.org/wiki/S%C3%ADmbolo_del_sistema_de_Windows}}). Consultado el 24 de mayo de 2022.

\bibitem{mit}{\it The MIT License - Open Source Initiative}. Disponible en {\it \url{https://opensource.org/licenses/MIT}}. Consultado el 29 de abril de 2022.
\end{thebibliography}