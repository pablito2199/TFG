\subsection{Tecnologías del backend}

El backend es lo que conecta la base de datos con el servidor, es decir, aquello con lo que el usuario no tiene una interacción directa. Entre las tecnologías que se han utilizado en esta parte, encontramos las siguientes:

\begin{itemize}
    \item {\bf Spring Boot} \cite{spring}: framework que permite la creación de aplicaciones en Java de una forma más sencilla. Spring simplifica el proceso de gestión de las dependencias, permitiendo que nos centremos lo máximo posible en el desarrollo de la aplicación.
    \item {\bf Gradle} \cite{gradle}: gestor de dependencias en aplicaciones de Java. Permite automatizar además el proceso de compilación, permitiendo configuraciones personalizadas y flexibles.
    \item {\bf MongoDB} \cite{mongodb}: base de datos NoSQL \cite{nosql}. Este tipo de bases de datos se caracteriza por estar orientadas hacia documentos, almacenando y recuperando datos en formatos que no son tablas.
    \item {\bf Swagger} \cite{swagger}: utilizado para crear la documentación de la API RESTful implementada en el servidor. Expone los servicios implementados en el servidor, permitiendo ver posibles valores a introducir, y conocer posibles respuestas en cada servicio.
\end{itemize}