\subsection{Tecnologías del frontend}

El {\it frontend} es la parte del sitio web con el que interactúan los usuarios, es decir, el lado del cliente. Entre las tecnologías que se han utilizado en esta parte, encontramos las siguientes:

\begin{itemize}
    \item {\bf React.js} \cite{react}: uno de los frameworks más populares de JavaScript. Permite el desarrollo de interfaces de usuario interactivas de forma sencilla.
    \item {\bf Tailwind} \cite{tailwind}: framework de CSS \cite{css} que permite un desarrollo ágil basado en clases de utilidad. Se ha utilizado para estilizar de forma rápida y sencilla la interfaz del usuario.
    \item {\bf Node.js} \cite{nodejs}: entorno de ejecución de JavaScript del proyecto.
    \item {\bf Npm} \cite{npm}: gestor de dependencias de los paquetes del cliente. Entre las principales dependencias utilizadas destacan:
        \begin{itemize}
        \item {\bf React Diff Viewer} \cite{reactdiffviewer}: componente de React que permite la visualización de las diferencias entre dos textos.
        \item {\bf Heroicons} \cite{heroicons}: iconografía que permite ser gestionada mediante propiedades CSS de Tailwind.
        \item {\bf Material UI} \cite{materialui}: componentes e iconografía que han sido implementados con el objetivo de un desarrollo más rápido en páginas web. También constan de integración con Tailwind.
        \end{itemize}
\end{itemize}