\subsection{Arquitectura global}

\begin{figure}[H]
\centerline{\includegraphics[width=9cm]{figuras/diseño/arquitecturaglobal.PNG}}
\caption{Arquitectura global del sistema.}
\label{enlaceArquitecturaGlobal}
\end{figure}

En la arquitectura global del sistema se pueden diferenciar tres componentes: el cliente, el servidor, y los almacenes de datos.
\\

El cliente está conformado por todos aquellos navegadores web que realizan llamadas a la API del servidor mediante su propia interfaz. Todo esto se ha realizado gracias a Node.js \cite{nodejs}, el entorno de ejecución de JavaScript del proyecto.
\\

Desde el cliente se realizan solicitudes HTTP hacia el servidor, que expone todos sus servicios en una API, la cual está documentada con Swagger \cite{swagger}. Con ello, el cliente intercambia información con el servidor según la llamada realizada: GET para obtener información, POST para crear objetos, PUT y PATCH para modificar información, y DELETE para eliminar objetos.
\\

La información que el servidor obtiene ha de recuperarla y almacenarla en algún sitio. En el caso de la API del DOG, solo recibe información de esta, obteniendo las leyes almacenadas en el servidor del DOG. Con respecto a la base de datos de MongoDB \cite{mongodb}, de esta puede obtener información, almacenar datos, modificar y eliminar.
\\

Con esta arquitectura se demuestra el cumplimiento de dos requisitos no funcionales descritos: el {\bf NFR-08: Uso de tecnolías web y servicios RESTful} (uso de Node.js) y el {\bf NFR-09: Base de datos NoSQL} (uso de MongoDB).