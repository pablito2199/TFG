\subsection{Arquitectura  del servidor}

El servidor se ha implementado con Spring Boot \cite{spring}, un framework que permite la creación de aplicaciones en Java de una forma más sencilla. En cuanto a la base de datos a la que se conecta, se trata de una base de datos NoSQL, donde los datos están orientados hacia el almacenamiento de documentos. En este caso se ha utilizado MongoDB.

\begin{figure}[H]
\centerline{\includegraphics[width=15cm]{figuras/diseño/FicherosServidor.png}}
\caption{Arquitectura del servidor.}
\label{enlaceArquitecturaServidor}
\end{figure}

El servidor está dividido en un conjunto de directorios, cada uno con una función diferente. 
\\

Todas las dependencias del servidor son definidas en el archivo {\bf build.gradle}. Dichas dependencias son tratadas por Gradle \cite{gradle}, un gestor de dependencias en aplicaciones de Java.
\\

En el subdirectorio {\bf main}, encontraremos una carpeta denominada {\bf resources}. En el fichero contenido en ella, se define el nombre de la base de datos que ha de utilizar la aplicación.
\\

Si accedemos al directorio {\bf java/tfg/project}, encontramos el contenido principal del servidor. El archivo {\bf Application} es el encargado de desplegar el servidor, pues contiene la función main del programa.
\\

En el directorio {\bf Config}, se localizan archivos de configuración del propio servidor, tales como aspectos de seguridad, documentación, privacidad, etc. 
\\

Otro directorio es el {\bf Filter}, donde se tratan distintos aspectos de seguridad con respecto a los usuarios que utilizan la aplicación, como puede ser el inicio de sesión, gestión de contraseñas o comprobación de roles.
\\

En la carpeta {\bf Model} encontramos los distintos objetos que conforman la base de la aplicación. Podemos destacar entre ellos el objeto {\bf FinalDocument}, que está formado por todos los atributos de la ley, y {\bf User}, que es el objeto utilizado para gestionar los usuarios. Todos estos objetos son empleados por las clases definidas en las carpetas {\bf Controller, Service y Repository}.
\\

En la carpeta {\bf Controller}, encontramos todos los archivos encargados de redirigir las llamadas a los servicios necesarios en la carpeta Service. Estos se encargan de localizar cualquier llamada al servidor (GET, POST, ...), y entregarla a los servicios.
\\

Con respecto al directorio {\bf Service}, este contiene los archivos donde se localizan los servicios web. Aquí se implementa la lógica de negocio de la aplicación, y no se conserva estado. Es llamada por los controladores, y solicita información a los repositorios donde se almacena toda la información.
\\

Como último directorio, se encuentra {\bf Repository}, donde se almacenan los ficheros que se encargan del acceso y almacenamiento de datos. Estos gestionan la conexión con la base de datos para poder obtener/almacenar información en ella.
\\

En resumen, las funcionalidades de los controladores, servicios y repositorios son las siguientes:
\begin{itemize}
    \item {\bf LocalController, FinalDocumentService y FinalDocumentRepository}: se encargan de gestionar las operaciones relacionadas con las leyes.
    \item {\bf UserController, UserService y UserRepository}: se encargan de gestionar las operaciones relacionadas con los usuarios.
    \item {\bf AuthController, AuthenticationService y UserRepository}: se encargan de gestionar las operaciones relacionadas con aspectos de seguridad respecto a los usuarios, como el inicio de sesión.
    \item {\bf XuntaController}: se encarga de gestionar las llamadas a la API de la Xunta.
\end{itemize}

\subsubsection{Documentación de la API del servidor}

La documentación de la API del servidor se ha realizado con Swagger \cite{swagger}, siendo posible consultarse en mayor detalle accediendo a la URI ``/swagger-ui/index.html''. Por ejemplo, en el caso de desplegar el servidor localmente, el enlace sería {\it \url{http://localhost:8080/swagger-ui/index.html}}. No obstante, se realiza aquí un pequeño resumen de las principales operaciones ofrecidas.

\begin{figure}[H]
\centerline{\includegraphics[width=15cm]{figuras/diseño/APIServidor.PNG}}
\caption{API del servidor.}
\label{enlaceAPIServidor}
\end{figure}

\begin{itemize}
    \item Para las operaciones del DOG se realiza un GET para obtener las normas buscadas. Es necesario estar autenticado. La URI de acceso es {\it /xunta}.
    \item Para los aspectos de autenticación no es necesario estar autenticado. La única operación existente es un POST para realizar el login. Se obtiene un token JWT \cite{jwt} para mayor seguridad (NFR-10). La URI de acceso es {\it /login}.
    \item Las operaciones relacionadas con leyes son: GET para obtener todos los datos, solo el sumario, el documento HTML de la ley y el id de la ley. POST se utiliza a la hora de almacenar nuevas leyes. PUT para modificar una ley completa. PATCH para modificar parcialmente atributos de una ley. DELETE para eliminar documentos. La URI de acceso general es {\it /local}.
    \item En cuanto a los usuarios, solo se permite realizar GET para obtener los datos de un usuario, y POST para crear nuevos usuarios. La URI de acceso general es {\it /users}.
\end{itemize}