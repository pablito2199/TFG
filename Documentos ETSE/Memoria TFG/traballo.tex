%
%  PARA TRABALLOS EN GALLEGO USAR (LINEA 12): \usepackage[galician]{babel}
%  PARA TRABALLOS EN CASTELLANO USAR (LINEA 13): \usepackage[spanish]{babel}
%
% Para los acentos usamos codificacion UTF-8 (LINEA 10): \usepackage[utf8]{inputenc} 
% Si se usase la codificacion es_ES.ISO-8859-1 (LINEA 11): \usepackage[latin1]{inputenc}
% La conversion de acentos se hace con: iconv -f UTF-8 -t ISO-8859-1 filename.tex
%
% Como se incluyen figuras eps hay que compilar con: latex traballo , dvipdf traballo
%

\documentclass[12pt,twoside,a4paper]{book}
% pódense engadir todos os packages necesarios
\usepackage[utf8]{inputenc}
% \usepackage[latin1]{inputenc}
\usepackage[spanish]{babel}
\usepackage{graphicx}
\usepackage[dvips]{epsfig}
\usepackage{amssymb}
\usepackage{eurosym}
\usepackage{float}
\usepackage{latexsym}
\usepackage{a4}
\usepackage{listings}
\usepackage{hyperref} % menús no pdf pero non leva ben co package galician


\addto\captionsgalician{\def\contentsname{Memoria tipo B -- \'{I}ndice xeral }}
	

\begin{document}
\pagestyle{empty}
\begin{center}
	{\bf\Large UNIVERSIDADE DE SANTIAGO DE COMPOSTELA}
	
	\vspace{0.5cm}
	\includegraphics[width=5cm]{figuras/logo_usc.eps}
	
	\vspace{0.5cm}
	{\bf\large ESCOLA TÉCNICA SUPERIOR DE ENXEÑARÍA}
	
	\vspace{3cm}
	{\bf\LARGE Herramienta para la creación y edición colaborativa de documentos}
	
	%\vspace{0.5cm}
	%{\bf\LARGE Subtítulo do Traballo de Fin de Grao}
\end{center}

\vspace{2cm}
\hspace{4cm}\begin{tabular}{l}
	{\it\Large Autor:} \\
	{\bf\Large Pablo Tarrío Otero} \\
	~ \\
	{\it\Large Tutores:} \\
	{\bf\Large Manuel Lama Penín} \\
	{\bf\Large Juan Carlos Vidal Aguiar} \\
	{\bf\Large Víctor José Gallego Fontenla} \\
\end{tabular}

\vspace{2cm}
\begin{center}
	{\bf\Large Grado en Ingeniería Informática}
	
	\vspace{0.5cm}
	{\bf\large Junio 2022}
	
	\vspace{0.5cm}
	Trabajo de Fin de Grado presentado en la {\it Escola Técnica Superior de Enxeñaría} de la {\it Universidade de Santiago de Compostela} para la obtención del Grado en Ingeniería Informática
\end{center}
\cleardoublepage
\pagestyle{plain}
\pagenumbering{roman}
\includegraphics[width=4cm]{figuras/logo_usc.eps}

\vspace{1cm}
{\bf D. Manuel Lama Penín}, Profesor del Departamento de Electrónica y Computación de la {\it Universidade de Santiago de Compostela}, {\bf D. Juan Carlos Vidal Aguiar}, Profesor del Departamento de Electrónica y Computación de la {\it Universidade de Santiago de Compostela}, y {\bf D. Víctor José Gallego Fontenla}, Investigador Predoctoral FPU del {\it Ministerio de Ciencia, Innovación y Universidades},

\vspace{1cm}
INFORMAN:

\vspace{1cm}
Que la presente memoria, titulada {\it Herramienta para la creación y edición
colaborativa de documentos}, presentada por {\bf D. Pablo Tarrío Otero} para superar los créditos correspondientes al Trabajo de Fin de Grado de la titulación de Grado en Ingeniería Informática, se realizó bajo nuestra tutoría en el Departamento de Electrónica y Computación de la {\it Universidade de Santiago de Compostela}.

\vspace{1cm}
Y para que así conste a los efectos oportunos, expiden el presente informe en Santiago de Compostela, a 21 de julio de 2022:

\vspace{0.5cm}
\begin{tabular}{ll}
	Tutor, & \hspace{3.6cm} Cotutor, \\
	~ \\
	~ \\
	~ \\
	~ \\
	Manuel Lama Penín & \hspace{3.6cm} Juan Carlos Vidal Aguiar
\end{tabular}

\vspace{1cm}
\begin{tabular}{ll}
	Cotutor, & \hspace{2cm} Alumno, \\
	~ \\
	~ \\
	~ \\
	~ \\
	Víctor José Gallego Fontenla & \hspace{2cm} Pablo Tarrío Otero
\end{tabular}

 % paxina de certificación (optativa)
\cleardoublepage
\pagestyle{plain}
\chapter*{Agradecimientos}
Se se quere pór algún agradecemento, este vai aquí.

 % paxina de agradecementos (optativa) 
\cleardoublepage
\pagestyle{plain}
\chapter*{Resumen}
Este trabajo de fin de grado aborda el desarrollo de una herramienta colaborativa que permitirá la creación, edición y consolidación de documentos por parte de varios usuarios. Esta herramienta será integrada en una plataforma de consulta de legislación consolidada gallega, facilitando la importación de legislación desde el Diario Oficial de Galicia (DOG) \cite{dog}, así como el mantenimiento con la aplicación en las sucesivas modificaciones que se vayan realizando.
\\

De esta forma, los usuarios podrán importar documentos a la herramienta colaborativa a partir de leyes importadas del DOG. Por una parte, se permite la búsqueda de leyes y normas encontradas en el DOG filtrando según los parámetros deseados. Adicionalmente, podrá verse una previsualización en la página oficial del DOG. Dichos documentos encontrados en el DOG podrán ser importados a la herramienta colaborativa, donde se podrá proceder a su edición. En cuanto a la edición, se ofrece la posibilidad de proponer cambios a la norma que se va a editar, manipular las leyes vinculadas a esta, así como escribir posibles notas aclaratorias. Además, todo ello será gestionado por usuarios previamente registrados. % páxina de resumo (optativa) 

\cleardoublepage
\pagestyle{plain}
\tableofcontents
\listoffigures
\listoftables

% Agora incluimos os capítulos. Cambiamos a numeración e as cabeceiras
\cleardoublepage
\pagenumbering{arabic}
\setcounter{page}{1}
\pagestyle{headings}
\section{Diagramas de secuencia}

En esta sección se describen los diagramas de secuencia de las operaciones más importantes realizadas en el sistema. De esta forma, se podrá observar de una forma más ilustrativa el funcionamiento de la aplicación, y las interacciones del usuario con esta última.
\cleardoublepage
\section{Diagramas de secuencia}

En esta sección se describen los diagramas de secuencia de las operaciones más importantes realizadas en el sistema. De esta forma, se podrá observar de una forma más ilustrativa el funcionamiento de la aplicación, y las interacciones del usuario con esta última.
\cleardoublepage
\section{Diagramas de secuencia}

En esta sección se describen los diagramas de secuencia de las operaciones más importantes realizadas en el sistema. De esta forma, se podrá observar de una forma más ilustrativa el funcionamiento de la aplicación, y las interacciones del usuario con esta última.
\cleardoublepage
\section{Diagramas de secuencia}

En esta sección se describen los diagramas de secuencia de las operaciones más importantes realizadas en el sistema. De esta forma, se podrá observar de una forma más ilustrativa el funcionamiento de la aplicación, y las interacciones del usuario con esta última.
\cleardoublepage
\section{Diagramas de secuencia}

En esta sección se describen los diagramas de secuencia de las operaciones más importantes realizadas en el sistema. De esta forma, se podrá observar de una forma más ilustrativa el funcionamiento de la aplicación, y las interacciones del usuario con esta última.
\cleardoublepage

% Aquí empezan os apéndices
\appendix
\cleardoublepage
\section{Diagramas de secuencia}

En esta sección se describen los diagramas de secuencia de las operaciones más importantes realizadas en el sistema. De esta forma, se podrá observar de una forma más ilustrativa el funcionamiento de la aplicación, y las interacciones del usuario con esta última.
\cleardoublepage
\section{Diagramas de secuencia}

En esta sección se describen los diagramas de secuencia de las operaciones más importantes realizadas en el sistema. De esta forma, se podrá observar de una forma más ilustrativa el funcionamiento de la aplicación, y las interacciones del usuario con esta última.
\cleardoublepage
\section{Diagramas de secuencia}

En esta sección se describen los diagramas de secuencia de las operaciones más importantes realizadas en el sistema. De esta forma, se podrá observar de una forma más ilustrativa el funcionamiento de la aplicación, y las interacciones del usuario con esta última.
\cleardoublepage
\section{Diagramas de secuencia}

En esta sección se describen los diagramas de secuencia de las operaciones más importantes realizadas en el sistema. De esta forma, se podrá observar de una forma más ilustrativa el funcionamiento de la aplicación, y las interacciones del usuario con esta última.
\cleardoublepage
\section{Diagramas de secuencia}

En esta sección se describen los diagramas de secuencia de las operaciones más importantes realizadas en el sistema. De esta forma, se podrá observar de una forma más ilustrativa el funcionamiento de la aplicación, y las interacciones del usuario con esta última.
\cleardoublepage
\section{Diagramas de secuencia}

En esta sección se describen los diagramas de secuencia de las operaciones más importantes realizadas en el sistema. De esta forma, se podrá observar de una forma más ilustrativa el funcionamiento de la aplicación, y las interacciones del usuario con esta última.

\end{document}
